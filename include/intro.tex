The goal of MyDas was to develop and test a range of assessment methods to establish Maximum Sustainable Yield (MSY) or proxy MSY reference points, across the spectrum of data-limited stocks. To achive this aim seven tasks were identified namely: i) Stock prioritisation, ii) Data Collation, iii) Framework Development; iv) Performance Appraisal, v) Reference Point Comparison, vi) Liaison with Marine Institute, and vii) Linkage with other Projects. 

Case studies were identified based on their economic and ecological importance and then datasets on life histories, commercial catches and surveys collated. Candidate assessment methods were identified and their performance and that of alternative reference points compared. As well as liason with the Marine Institute under the project also supported participation at the ICES Workshop On The Development Of Quantitative Assessment Methodologies Based On Life-History Traits, Exploitation Characteristics, And Other Relevant Parameters For Data-Limited Stocks (WKLIFE).

Outputs from MyDas were 
\begin{itemize}[noitemsep,topsep=0pt,parsep=0pt,partopsep=0pt]
 \item A collection of existing and new assessment models for data-limited stocks. 
 \item Proposed reference points for a range of stocks with associated management strategy evaluations to contribute to sustainable management of these stocks.
 \item Working documsents describing the methods and findings.
 \item Publication in peer-reviewed journals on new methods/tools/evaluations.
\end{itemize}

