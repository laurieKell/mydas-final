\documentclass[12pt,doublespacing,a4paper]{ouparticle}

%\usepackage[scaled]{helvet}
%\renewcommand\familydefault{\sfdefault} 
%\usepackage[T1]{fontenc}


\usepackage[authoryear]{natbib}
\usepackage{lineno}
\usepackage{pdfpages}
\usepackage{lscape}


\begin{document}

The use of trends in an index without a reference level (ToR iv) were explored using methods developed under the mydas project.  To do this Management Strategy Evaluation (MSE) was conducted to evaluate an empirical harvest control rule (HCR) based on a trend in an index of abundance. 

The  Operating Model (OM) was conditioning on turbot life history characteristics and the HCR was based on that used by the Commission for the Conservation of Southern Bluefin Tuna (CCSBT). The HCR has several parameters that require tuning (Hillary et al., 2016). When tuning a HCR the parameters are found by choosing values that best meet the objectives of  asset and stakeholders, i.e. optimises the outcomes modelled as a reward function. 

The HCR was modelled as part of a  Management Procedure (MP) where catches are increased when the trend in an index of abundance is positive, and decreased if the trend is negative, namely

where  is the slope in the regression of  against year for the most recent n years and  and  are  the tunable parameters andactions asymmetry so that decreases in the index do not result in the same relative change as as an increase.
When tuning an empirical MP it is run for a range of control parameters values (i.e. for,and ). These are then chosen based on the performance of the MP, i.e.  maximising a reward function based on management objectives. It can be difficult, however, to specify a single reward function, due to trade-offs between multiple objectives. Deciding which is the “best” MP therefore requires an iterative process involving managers, asset holders, stakeholders and scientists.
Once objectives are agreed the traditional way to find the control parameters is to perform  a grid search, i.e. an exhaustive search through a manually specified set of control parameters. Even for a limited number of control parameters this can take a substantial amount of computing time. Tuning was performed using random search where control parameters are selected from all the potential combinations at random.  Random search has proven to yield better results in comparison to grid search. Drawbacks of random search are that it may yields high variance during computing and since the selection of parameters is completely random no intelligence is used to sample the combinations and so lick plays its part. 

Trade offs between multiple objectives were evaluated by identifying pareto-optimal solutions (Mishra, et al., 2002) using support vector regression (SVR, Smola and Schölkopf, 2004). The best HCR parameters were then identified using a Genetic Algorithm (GA, Whitley, 1994). Both SVR and GAs are machine learning techniques. 
In optimisation studies with multi-objectives the focus is usually on finding a global optimum, i.e. the global Pareto-optimal frontier, representing the best possible objective values (Deb and Gupta, 2005).  However, in fisheries  there is usually high uncertainty about resource dynamics and solutions are therefore sensitive to the assumptions and environmental variability.  Therefore rather than finding global solution it is more important to find  robust solutions which are insensitive to uncertainty about processes. 
Figure 1 shows the trade-off between yield (Yield:MSY) and safety (the minimum expected recruitment relative to). Individual MSE (blue) results are highly variable due to variablity in recruitment and the Index of abundance used in the MP. The pareto frontier   derived from SVR are shown (red) and an example of an optimal solution highlighted (large dot).
Figures 2 show the calibration curves, obtained using the GA for the control parameters  and . This was obtained from the pareto frontiers by finding the values that corresponded to the optimal solution. If the management objectives are agreed the corresponding control value can be read off from the Y-axes. The scatter of points reflects that the Pareto frontiers are hyper-dimensional surfaces projected into 2 dimensions.
Once the control parameters that best met the management objectives were found the MSE was run for the control parameters for 2 scenarios corresponding to the Index of abundance CV (10%, 20% and 30%) and the number of years (3, 5, and 7 column) used in the regression to estimate the trend in the index; the summary statistics are shown in Figure 3.
An objective of the approach was to develop a risk based framework for conducting MSE, by allowing asset and stake holders to more easily to evaluate the trade-offs between management objectives and the impact of uncertainty when conducting MSE. The framework also provides an efficient way of tuning Management Procedures so that case specific management strategies can more easily be developed. However, since random search was used the outcomes partly depend on chance, the next step is to add intelligence by using machine learning to choose the control parameters.

The approached used demonstrates a potential stepwise procedure for conducting MSE namely
        ◦ First a single MSE is run using random search and the Pareto frontiers found.
        ◦ Objectives can be elicited from asset and stakeholders, and the trade-offs between them evaluated.
        ◦ Using the Pareto frontiers the control parameters can be derived by calibration.
        ◦ Next a set of robustness trials, can be developed for an agreed set of OMs that reflect the main uncertainties and the corresponding Pareto frontiers derived.
        ◦ A final set of control parameters can then be agreed following dialogue with asset and stakeholders




Figure 1. The trade-off between yield (Yield:MSY) and the average SSB relative to  are shown for the individual management strategy evaluations (blue) along with the pareto frontier (red).

Figure 2. Calibration regression values for the control parameters K1 and K2 for the pareto frontier for , large point is for safety~0.7.


Figure 3. Summary statistics from MSE.
References

Deb, K. and Gupta, H., 2005, March. Searching for robust Pareto-optimal solutions in multi-objective optimization. In International Conference on Evolutionary Multi-Criterion Optimization (pp. 150-164). Springer, Berlin, Heidelberg.

Hillary, R.M., Preece, A.L., Davies, C.R., Kurota, H., Sakai, O., Itoh, T., Parma, A.M., Butterworth, D.S., Ianelli, J. and Branch, T.A., 2016. A scientific alternative to moratoria for rebuilding depleted international tuna stocks. Fish and fisheries, 17(2), pp.469-482.

Smola, A.J. and Schölkopf, B., 2004. A tutorial on support vector regression. Statistics and computing, 14(3), pp.199-222.

Whitley, D., 1994. A genetic algorithm tutorial. Statistics and computing, 4(2), pp.65-85.

.

\title{An Evaluation of the Robustness of Length Based Indicators using Receiver Operating Characteristics.}

\author{%
\name{Laurence T. Kell}
\address{Centre for Environmental Policy, Imperial College London, London SW7 1NE}
\email{e-mail Lauriel@seapluslus.co.uk}
\and
\name{Co\'il\'in Minto}
\address{GMIT}
\email{Hans Gerritsen}
\address{Marine Institute}
}

\abstract{
%Scientific management advice for data rich stocks is based upon biological points and expert opinion to fix or meta-analyses to develop priors for difficult to estimate parameters. Whilst advice for data poor stocks is based upon life history parameters and outputs from data rich stock assessments to develop priors and propose indicators based upon life history traits.
%To help propose and develop robust and non-redundant productivity indicators the reliability and stability of the indices are evaluated with reference to the data rich set. 
}

\date{\today}

\keywords{}
 
\maketitle

\newpage


\linenumbers
\linespread{2}

\section{Outline}
\begin{itemize}
 \item Robustness, despite what you dont know
 \item Value-of-information
 \item Value-of-control
 \item Trends v. absolute
 \item Indicator for life histories
 \item Indicator for fisheries
\end{itemize}


\section{Introduction}

A substantial fraction of world fisheries are conducted on resources for which data are insufficient to conduct formal stock assessments and so status, productivity, and exploitation levels of many stocks and species are largely unknown \citep{thorson2015introduction}. Such stocks include by-caught, threatened, endangered, and protected species and are termed data-poor, data-limited, or information-poor. The adoption of the precautionary approach \citep[PA,][]{garcia1996precautionary}, however, still requires management plans and biological reference points for all stocks not just for the main commercial stocks where analytical assessments are available \citep{sainsbury2003ref}. 

Reference points are used in management plans as targets to maximise surplus production and as limits to minimise the risk of depleting a resource to a level where productivity is compromised. They must integrate dynamic processes such as growth, recruitment, mortality and connectivity into indices for exploitation level and spawning reproductive potential \citep{kell2015spawning}. In data poor situations life history parameters, such as maximum size and size at first maturity have been used as proxies for productivity \citep{roff1984evolution,jensen1996beverton,caddy1998short,reynolds2001life,denney2002life}. For example ICES uses indicators and proxy maximum sustainable yield (MSY) reference points based on length (e.g. the von Bertalanffy growth parameter $L_{\infty}$ and length at maturity $L_{mat}$) as part of a Precautionary Approach for stocks where only trends in mean length are available \citep[][]{ref}. 

An important characteristic of PA advice is that it should be robust to uncertainty. Where a robust control system is one that still functions correctly despite the presence of uncertainty \citep{radatz1990ieee, zhou1996robust}, while to be robust an indicator should be both reliable and stable. An indicator has high reliability if despite uncertainty it provides an accurate result, and it is stable if despite random error, similar results are produced across multiple trials. To evaluate the robustness of length based indicators and $MSY$ proxies  a simulation and sampling (i.e. sim-sam) procedure was employed where an Operatating Model (OM) was conditioned on life history parameters then psuedo data simulated using an Observation Error Model (OEM), allowing the indicators to be compared to the actual state of the resource.


\section{Material and Methods}

OMs were conditioned on life histories representing a range of population dynaics, namely turbot, brill, thornback ray, pollack and sprat. Scenarios that represent the main sources of uncertainty were also developed. Resource dynamics were simulated assuming an increasing trend in fishing mortality ($F$) that led to overfishing, then a recovery plan was implemented to bring fishing back to the $F_{MSY}$ level. Length based indicators were then generates using the OEM and compared to the fishing mortality in the OM, using Reciever Operating Characteristic (ROC) curves \citep{green1966signal}. A ROC curve can be thought of as a plot of the power as a function of the Type I Error of the decision rule. If the probability distributions for both detection and false alarm are known, the ROC curve is generated by plotting the cumulative distribution function (area under the probability distribution from  to the discrimination threshold) of the detection probability in the y-axis versus the cumulative distribution function of the false-alarm probability on the x-axis. A ROC analysis therefore provides a tool to select the best candidate indicators. 

\subsection{Methods}

For each species the life history parameters (\textbf{Table} \ref{tab:lh}) were used to parameterise a \cite{vonbert1957quantitative} growth curve, proportion mature-at-age, natural mortality \citep{lorenzen2002density} and a \cite{beverton1993dynamics} stock recruit relationship. Spawning stock biomass (SSB) was was used as a proxy for stock reproductive potential \citep[SRP][]{trippel_estimation_1999}. This assumes that fecundity is proportional to the mass-at-age of the sexually mature portion of the population irrespective of the demographic composition of adults \citep{murawski_impacts_2001} and that processes such as sexual maturity are simple functions of age \citep{matsuda_inconsistency_1996} and independent of gender.

These processes allow an equilibrium per-recruit model to be parameterised, which when combined with a stock recruitment relationship \cite{sissenwine1987alternative} is then used to condition a forward projection model to simulate the time series.

\subsubsection{Scenarios}

Even for data-rich stock assessments there is often large uncertainty about the dynamics \citep[i.e. model uncertainty;][]{punt2008refocusing}. This means that estimates of stock status are highly sensitive to assumptions about natural mortality-at-age \citep{jiao2012modelling}, vulnerability of age classes to the fisheries \citep{brooks2009analytical}, and the relationship between stock and recruitment which is difficult to estimate in practice \citep[e.g.][]{vert2013frequency,szuwalski2014examining,cury2014resolving,kell2015spawning,pepin2015reconsidering}. Therefore scenarios \citep[][]{ono2015importance,kell2015spawning,boorman1997recognising} representing these sources of uncertainty were developed for selection pattern, natural mortality, steepness of the stock recruitment relationship, recruitment variablility and sample size (\textbf{Table} \ref{tab:scen})

\begin{itemize}
 \item Selection pattern: \textbf{same as matirity ogive}; dome shaped; constant across all ages.
 \item Natural Mortality: \textbf{varies by age}; 0.2 
 \item Steepness of the stock recruitment: \textbf{0.7}; 0.9 
 \item Recruitment CV: \textbf{30\%}; \textbf{50\%}; \textbf{AR}
 \item Sample Size:\textbf{500}, \textbf{250}
\end{itemize}


\subsubsection{Indicators}


%\cite{punt2011among}


Simple catch rules have been developed for data poor stock, for example the “2 over 3” rule  of ICES (2012) which aims to keep stocks at their current level by multiplying recent catches by the trend in a biomass index. Catch rules that make use of more data sources have subsequently been developed (e.g. Fischer et al., submitted), using a rule of the form

\begin{equation} C_{y+1} = C_{y-1}rfb \end{equation}

The advised catch is based on the previous catch $C_{y-1}$, multiplied by three components $r$, $f$ and $b$, each representing a particular stock characteristic. Component $r$ corresponds to the trend in a biomass index, component $f$ uses a proxy  for the rateio $F:F_{MSY}$ based on length, and $b$ is a safeguard which protects the stock by not allowing catch to increase unlimitedly once the biomass index drops below a threshold. In this study we focus on the $f$ component, although the same approach can be used to screen any data source or even outputs from data rich assessments.

Empirical indicators based on length, used as proxies for exploitation rate, include

\begin{itemize}
 \item $L_{max5\%}$ mean length of largest 5\%
 \item $L_{95\%}$ $95^{th}$ percentile
 \item $P_{mega}$ Proportion of individuals above $L_{opt} + 10\%$
 \item $L_{25\%}$ $25^{th}$ percentile of length distribution
 \item $L_{c}$ Length at $50\%$ of modal abundance
 \item $L_{mean}$ Mean length of individuals $> L_c$
 \item $L_{max_{y}}$ Length class with maximum biomass in catch
 \item $L_{mean}$ Meanlength of individuals $> L$
\end{itemize}

and potential reference points include

\begin{itemize}
 \item $L_{\infty}$
 \item $L_{mat}$
 \item $L_{opt} = L_{\infty}\frac{3}{3+\frac{M}{K}}$, assuming $M/K = 1.5$ gives $\frac{2}{3}L_{\infty}$
 \item $L_{F=M} =  0,75l_c+0.25l_{\infty}$
\end{itemize}


\begin{landscape}

\begin{center}
\begin{table}
\renewcommand{\arraystretch}{0.75}
        \begin{tabular}{ | l | p{5cm} | p{2.5cm} | p{2cm}| p{2cm}| p{3cm} |}
 
        \hline
        Indicator & Calculation & Reference point & Indicator ratio & Expected value & Property\\ \hline
	$L_{max5$ & Mean length of largest 5\%  & $L_{\infty}$ & $Lmax5\% / L_{\infty}$ & $> 0.8$ & Conservation (large individuals) \\
        \hline
        $L_{max95\%}$ & $95^{th}$ percentile & $L_{\infty}$ & $L95\% / L_{\infty}$ & $> 0.8$ & Conservation (large individuals) \\
        \hline
	$L_{25\%}$ & $25^{th}$ percentile of length distribution & $L_{mat}$ & $L25\% / L_{mat}$ & $> 1$ & Conservation (immatures) \\
        \hline
	$L_c$ & Length at 50\% of modal abundance & $L_{mat}$ & $L_c/L_{mat}$ & $>1$ & Conservation (immatures) \\
        \hline
	$L_{mean}$ & Mean length of individuals  $> L_c$ & $L_{opt}$ = 2/3  $L_{\infty}$ & $L_{mean}/L_{opt}$ & $≈ 1$ & Optimal yield \\
        \hline
	$L_{maxY}$ & Length class with maximum biomass in catch & $L_{opt}$ = 2/3  $L_{\infty}$ &$L_{maxY} / L_{opt}$ & $=1$ & Optimal yield \\
        \hline
	$L_{mean}$ & Mean length of individuals $> L_c$ & $L_{opt}$ = 2/3  $L_{\infty}$ &$L_{mean} / L_{F=M}$ & $\geq 1$ & MSY \\
        \hline
	$L_{bar}$ & Mean length & $L_{F=M} = 0.75L_c + 0.25L_{\infty}$ & $L_{mat}$ & $> 1$ &  \\
        \hline
	$P_{mega}$ & Proportion of individuals above $L_{opt}$ + 10\%, $L_{opt}$ is estimated from $L_{\infty}$. &  & $P_{mega}$ & $> 0.3$ & Conservation (large individuals) \\
        \hline
    \end{tabular}
\end{table}    
\end{center}

\end{landscape}



\subsubsection{Receiver Operating Characteristics}

Indicators and reference point may be biased and have poor precision due to uncertainty about life history parameters, lags between size distribution and exploitation levels, variability in recruitment and resonant cohort effects that can produce long term fluctuation in the time series \citep{botsford2014cohort,bjoernstad2004trends}. This means that the reference level that can best identify the system state is unlikely to be $L_{mean}/L_{F=M}$ but a multiple of it, i.e. the discrimination threshold. The discrimination threshold is the value of $L_{mean}/L_{F=M}$ at which the a stock is said to be undergoing overfishing. A ROC curve plots the true positive rate (TPR) against the false positive rate  (FPR) at various threshold settings. Risks are asymmetric, i.e. the risk of indicating overfishing is occurring when the stock is sustainably exploited is not the same as the risk of failing to identify overfishing, and so it may be desirable to adjust the threshold to increase or decrease the sensitivity to false positives.

The ROC curve is constructed by sorting the observed outcomes from the OM by their predicted scores (from the OEM) with the highest scores first. The cumulative True Positive Rate (TPR) and True Negative Rate (TNR) are then calculated for the ordered observed outcomes. The ROC curve can also be thought of as a plot of the power as a function of the Type I Error of the decision rule. If the probability distributions for both detection and false alarm are known, the ROC curve can be generated by plotting the cumulative distribution function (area under the probability distribution from  to the discrimination threshold) of the detection probability in the y-axis versus the cumulative distribution function of the false-alarm probability on the x-axis. A ROC analysis therefore provides a tool to select the best candidate indicators. 


\section{Results}


\textbf{Figure} \ref{fig:lh} Life history parameters

\textbf{Figure} \ref{fig:om} shows the simulated exploitation history relative to $MSY$ reference points for each of the stocks for the base case. Fishing was initially low then increased to twice $F_{MSY}$, following which a recovery plan was implemented in order to reduce F to $F_{MSY}$. The coloured regions indicate exploitation levels.

\textbf{Figure} \ref{fig:lfd} Example of simulated length samples 

\textbf{Figure} \ref{fig:indicators} Example indicators with worm plots

\textbf{Figure} \ref{fig:roc} ROC curves for ICES indicators

\textbf{Figure} \ref{fig:discrim} boxplots comparing ICES indicators/reference points to ROC tresholds

\textbf{Figure} \ref{fig:tree} Regression Tree


\section{Discussion}


\begin{description}
 \item[Uncertainty]  
 \item[Risk]     
 \item[Management Frameworks] 
 \item[Robustness]
 \item[Lessons for data poor case studies]  
 \item[Lessons for data rich case studies] 
\end{description}

%For risks to be managed in a consistent way given the range of uncertainties across data rich and poor stocks requires OMs to be condition on appropriate processes. This allows an evaluation of the relative value-of-information and the value-of-control. In the former case this involves demonstrating the benefit of obtain better knowledge and data, i.e. to move a stock between categories, and in the later to consider alternative forms of indicators and control rules to develop robust advice. ROC curves are related in a direct and natural way to a cost/benefit analysis of diagnostic decision making, i.e. can be used to identify the value of control

%Setting of appropriate reference levels in the f and b component of the rules, and the extent to which this could be done with tuning that depends on life-history traits and/or the nature of the time-series was addressed using tools developed under the MyDas project. The aim of MyDas is to develop and test a range of assessment models and methods to establish Maximum Sustainable Yield (MSY), or proxy MSY reference points across the spectrum of data-limited stocks. To tune a catch rule of the form r f b, requires selecting indicators and reference points for each of the r and b components and then finding multipliers and thresholds for the component in order to combine them into a single rule. Doing this on a stock specific basis can take considerable time using MSE alone, especially as a variety of indicators and reference points can be used.  Therefore an example was developed using a Receiver Operating Characteristic or ROC curve, to show how potential indicators and reference points can be screened for a range of uncertainties before conducting MSE. 


\section{Conclusions}

\begin{itemize}
 \item 
 \item 
 \item 
 \item 
\end{itemize}

\section{Acknowledgement}

Laurence Kell's involvement was funded through the MyDas project under the Marine Biodiversity Scheme which is financed by the Irish government and the European Maritime and Fisheries Fund (EMFF) as part of the EMFF Operational Programme for 2014-2020. 

\clearpage
%\bibliography{/home/laurence-kell/Desktop/projects/mydas/papers/refs.bib}
%\bibliographystyle{apalike}
\documentclass[12pt,doublespacing,a4paper]{ouparticle}

%\usepackage[scaled]{helvet}
%\renewcommand\familydefault{\sfdefault} 
%\usepackage[T1]{fontenc}


\usepackage[authoryear]{natbib}
\usepackage{lineno}
\usepackage{pdfpages}
\usepackage{lscape}


\begin{document}

The use of trends in an index without a reference level (ToR iv) were explored using methods developed under the mydas project.  To do this Management Strategy Evaluation (MSE) was conducted to evaluate an empirical harvest control rule (HCR) based on a trend in an index of abundance. 

The  Operating Model (OM) was conditioning on turbot life history characteristics and the HCR was based on that used by the Commission for the Conservation of Southern Bluefin Tuna (CCSBT). The HCR has several parameters that require tuning (Hillary et al., 2016). When tuning a HCR the parameters are found by choosing values that best meet the objectives of  asset and stakeholders, i.e. optimises the outcomes modelled as a reward function. 

The HCR was modelled as part of a  Management Procedure (MP) where catches are increased when the trend in an index of abundance is positive, and decreased if the trend is negative, namely

where  is the slope in the regression of  against year for the most recent n years and  and  are  the tunable parameters andactions asymmetry so that decreases in the index do not result in the same relative change as as an increase.
When tuning an empirical MP it is run for a range of control parameters values (i.e. for,and ). These are then chosen based on the performance of the MP, i.e.  maximising a reward function based on management objectives. It can be difficult, however, to specify a single reward function, due to trade-offs between multiple objectives. Deciding which is the “best” MP therefore requires an iterative process involving managers, asset holders, stakeholders and scientists.
Once objectives are agreed the traditional way to find the control parameters is to perform  a grid search, i.e. an exhaustive search through a manually specified set of control parameters. Even for a limited number of control parameters this can take a substantial amount of computing time. Tuning was performed using random search where control parameters are selected from all the potential combinations at random.  Random search has proven to yield better results in comparison to grid search. Drawbacks of random search are that it may yields high variance during computing and since the selection of parameters is completely random no intelligence is used to sample the combinations and so lick plays its part. 

Trade offs between multiple objectives were evaluated by identifying pareto-optimal solutions (Mishra, et al., 2002) using support vector regression (SVR, Smola and Schölkopf, 2004). The best HCR parameters were then identified using a Genetic Algorithm (GA, Whitley, 1994). Both SVR and GAs are machine learning techniques. 
In optimisation studies with multi-objectives the focus is usually on finding a global optimum, i.e. the global Pareto-optimal frontier, representing the best possible objective values (Deb and Gupta, 2005).  However, in fisheries  there is usually high uncertainty about resource dynamics and solutions are therefore sensitive to the assumptions and environmental variability.  Therefore rather than finding global solution it is more important to find  robust solutions which are insensitive to uncertainty about processes. 
Figure 1 shows the trade-off between yield (Yield:MSY) and safety (the minimum expected recruitment relative to). Individual MSE (blue) results are highly variable due to variablity in recruitment and the Index of abundance used in the MP. The pareto frontier   derived from SVR are shown (red) and an example of an optimal solution highlighted (large dot).
Figures 2 show the calibration curves, obtained using the GA for the control parameters  and . This was obtained from the pareto frontiers by finding the values that corresponded to the optimal solution. If the management objectives are agreed the corresponding control value can be read off from the Y-axes. The scatter of points reflects that the Pareto frontiers are hyper-dimensional surfaces projected into 2 dimensions.
Once the control parameters that best met the management objectives were found the MSE was run for the control parameters for 2 scenarios corresponding to the Index of abundance CV (10%, 20% and 30%) and the number of years (3, 5, and 7 column) used in the regression to estimate the trend in the index; the summary statistics are shown in Figure 3.
An objective of the approach was to develop a risk based framework for conducting MSE, by allowing asset and stake holders to more easily to evaluate the trade-offs between management objectives and the impact of uncertainty when conducting MSE. The framework also provides an efficient way of tuning Management Procedures so that case specific management strategies can more easily be developed. However, since random search was used the outcomes partly depend on chance, the next step is to add intelligence by using machine learning to choose the control parameters.

The approached used demonstrates a potential stepwise procedure for conducting MSE namely
        ◦ First a single MSE is run using random search and the Pareto frontiers found.
        ◦ Objectives can be elicited from asset and stakeholders, and the trade-offs between them evaluated.
        ◦ Using the Pareto frontiers the control parameters can be derived by calibration.
        ◦ Next a set of robustness trials, can be developed for an agreed set of OMs that reflect the main uncertainties and the corresponding Pareto frontiers derived.
        ◦ A final set of control parameters can then be agreed following dialogue with asset and stakeholders




Figure 1. The trade-off between yield (Yield:MSY) and the average SSB relative to  are shown for the individual management strategy evaluations (blue) along with the pareto frontier (red).

Figure 2. Calibration regression values for the control parameters K1 and K2 for the pareto frontier for , large point is for safety~0.7.


Figure 3. Summary statistics from MSE.
References

Deb, K. and Gupta, H., 2005, March. Searching for robust Pareto-optimal solutions in multi-objective optimization. In International Conference on Evolutionary Multi-Criterion Optimization (pp. 150-164). Springer, Berlin, Heidelberg.

Hillary, R.M., Preece, A.L., Davies, C.R., Kurota, H., Sakai, O., Itoh, T., Parma, A.M., Butterworth, D.S., Ianelli, J. and Branch, T.A., 2016. A scientific alternative to moratoria for rebuilding depleted international tuna stocks. Fish and fisheries, 17(2), pp.469-482.

Smola, A.J. and Schölkopf, B., 2004. A tutorial on support vector regression. Statistics and computing, 14(3), pp.199-222.

Whitley, D., 1994. A genetic algorithm tutorial. Statistics and computing, 4(2), pp.65-85.

.

\title{An Evaluation of the Robustness of Length Based Indicators using Receiver Operating Characteristics.}

\author{%
\name{Laurence T. Kell}
\address{Centre for Environmental Policy, Imperial College London, London SW7 1NE}
\email{e-mail Lauriel@seapluslus.co.uk}
\and
\name{Co\'il\'in Minto}
\address{GMIT}
\email{Hans Gerritsen}
\address{Marine Institute}
}

\abstract{
%Scientific management advice for data rich stocks is based upon biological points and expert opinion to fix or meta-analyses to develop priors for difficult to estimate parameters. Whilst advice for data poor stocks is based upon life history parameters and outputs from data rich stock assessments to develop priors and propose indicators based upon life history traits.
%To help propose and develop robust and non-redundant productivity indicators the reliability and stability of the indices are evaluated with reference to the data rich set. 
}

\date{\today}

\keywords{}
 
\maketitle

\newpage


\linenumbers
\linespread{2}

\section{Outline}
\begin{itemize}
 \item Robustness, despite what you dont know
 \item Value-of-information
 \item Value-of-control
 \item Trends v. absolute
 \item Indicator for life histories
 \item Indicator for fisheries
\end{itemize}


\section{Introduction}

A substantial fraction of world fisheries are conducted on resources for which data are insufficient to conduct formal stock assessments and so status, productivity, and exploitation levels of many stocks and species are largely unknown \citep{thorson2015introduction}. Such stocks include by-caught, threatened, endangered, and protected species and are termed data-poor, data-limited, or information-poor. The adoption of the precautionary approach \citep[PA,][]{garcia1996precautionary}, however, still requires management plans and biological reference points for all stocks not just for the main commercial stocks where analytical assessments are available \citep{sainsbury2003ref}. 

Reference points are used in management plans as targets to maximise surplus production and as limits to minimise the risk of depleting a resource to a level where productivity is compromised. They must integrate dynamic processes such as growth, recruitment, mortality and connectivity into indices for exploitation level and spawning reproductive potential \citep{kell2015spawning}. In data poor situations life history parameters, such as maximum size and size at first maturity have been used as proxies for productivity \citep{roff1984evolution,jensen1996beverton,caddy1998short,reynolds2001life,denney2002life}. For example ICES uses indicators and proxy maximum sustainable yield (MSY) reference points based on length (e.g. the von Bertalanffy growth parameter $L_{\infty}$ and length at maturity $L_{mat}$) as part of a Precautionary Approach for stocks where only trends in mean length are available \citep[][]{ref}. 

An important characteristic of PA advice is that it should be robust to uncertainty. Where a robust control system is one that still functions correctly despite the presence of uncertainty \citep{radatz1990ieee, zhou1996robust}, while to be robust an indicator should be both reliable and stable. An indicator has high reliability if despite uncertainty it provides an accurate result, and it is stable if despite random error, similar results are produced across multiple trials. To evaluate the robustness of length based indicators and $MSY$ proxies  a simulation and sampling (i.e. sim-sam) procedure was employed where an Operatating Model (OM) was conditioned on life history parameters then psuedo data simulated using an Observation Error Model (OEM), allowing the indicators to be compared to the actual state of the resource.


\section{Material and Methods}

OMs were conditioned on life histories representing a range of population dynaics, namely turbot, brill, thornback ray, pollack and sprat. Scenarios that represent the main sources of uncertainty were also developed. Resource dynamics were simulated assuming an increasing trend in fishing mortality ($F$) that led to overfishing, then a recovery plan was implemented to bring fishing back to the $F_{MSY}$ level. Length based indicators were then generates using the OEM and compared to the fishing mortality in the OM, using Reciever Operating Characteristic (ROC) curves \citep{green1966signal}. A ROC curve can be thought of as a plot of the power as a function of the Type I Error of the decision rule. If the probability distributions for both detection and false alarm are known, the ROC curve is generated by plotting the cumulative distribution function (area under the probability distribution from  to the discrimination threshold) of the detection probability in the y-axis versus the cumulative distribution function of the false-alarm probability on the x-axis. A ROC analysis therefore provides a tool to select the best candidate indicators. 

\subsection{Methods}

For each species the life history parameters (\textbf{Table} \ref{tab:lh}) were used to parameterise a \cite{vonbert1957quantitative} growth curve, proportion mature-at-age, natural mortality \citep{lorenzen2002density} and a \cite{beverton1993dynamics} stock recruit relationship. Spawning stock biomass (SSB) was was used as a proxy for stock reproductive potential \citep[SRP][]{trippel_estimation_1999}. This assumes that fecundity is proportional to the mass-at-age of the sexually mature portion of the population irrespective of the demographic composition of adults \citep{murawski_impacts_2001} and that processes such as sexual maturity are simple functions of age \citep{matsuda_inconsistency_1996} and independent of gender.

These processes allow an equilibrium per-recruit model to be parameterised, which when combined with a stock recruitment relationship \cite{sissenwine1987alternative} is then used to condition a forward projection model to simulate the time series.

\subsubsection{Scenarios}

Even for data-rich stock assessments there is often large uncertainty about the dynamics \citep[i.e. model uncertainty;][]{punt2008refocusing}. This means that estimates of stock status are highly sensitive to assumptions about natural mortality-at-age \citep{jiao2012modelling}, vulnerability of age classes to the fisheries \citep{brooks2009analytical}, and the relationship between stock and recruitment which is difficult to estimate in practice \citep[e.g.][]{vert2013frequency,szuwalski2014examining,cury2014resolving,kell2015spawning,pepin2015reconsidering}. Therefore scenarios \citep[][]{ono2015importance,kell2015spawning,boorman1997recognising} representing these sources of uncertainty were developed for selection pattern, natural mortality, steepness of the stock recruitment relationship, recruitment variablility and sample size (\textbf{Table} \ref{tab:scen})

\begin{itemize}
 \item Selection pattern: \textbf{same as matirity ogive}; dome shaped; constant across all ages.
 \item Natural Mortality: \textbf{varies by age}; 0.2 
 \item Steepness of the stock recruitment: \textbf{0.7}; 0.9 
 \item Recruitment CV: \textbf{30\%}; \textbf{50\%}; \textbf{AR}
 \item Sample Size:\textbf{500}, \textbf{250}
\end{itemize}


\subsubsection{Indicators}


%\cite{punt2011among}


Simple catch rules have been developed for data poor stock, for example the “2 over 3” rule  of ICES (2012) which aims to keep stocks at their current level by multiplying recent catches by the trend in a biomass index. Catch rules that make use of more data sources have subsequently been developed (e.g. Fischer et al., submitted), using a rule of the form

\begin{equation} C_{y+1} = C_{y-1}rfb \end{equation}

The advised catch is based on the previous catch $C_{y-1}$, multiplied by three components $r$, $f$ and $b$, each representing a particular stock characteristic. Component $r$ corresponds to the trend in a biomass index, component $f$ uses a proxy  for the rateio $F:F_{MSY}$ based on length, and $b$ is a safeguard which protects the stock by not allowing catch to increase unlimitedly once the biomass index drops below a threshold. In this study we focus on the $f$ component, although the same approach can be used to screen any data source or even outputs from data rich assessments.

Empirical indicators based on length, used as proxies for exploitation rate, include

\begin{itemize}
 \item $L_{max5\%}$ mean length of largest 5\%
 \item $L_{95\%}$ $95^{th}$ percentile
 \item $P_{mega}$ Proportion of individuals above $L_{opt} + 10\%$
 \item $L_{25\%}$ $25^{th}$ percentile of length distribution
 \item $L_{c}$ Length at $50\%$ of modal abundance
 \item $L_{mean}$ Mean length of individuals $> L_c$
 \item $L_{max_{y}}$ Length class with maximum biomass in catch
 \item $L_{mean}$ Meanlength of individuals $> L$
\end{itemize}

and potential reference points include

\begin{itemize}
 \item $L_{\infty}$
 \item $L_{mat}$
 \item $L_{opt} = L_{\infty}\frac{3}{3+\frac{M}{K}}$, assuming $M/K = 1.5$ gives $\frac{2}{3}L_{\infty}$
 \item $L_{F=M} =  0,75l_c+0.25l_{\infty}$
\end{itemize}


\begin{landscape}

\begin{center}
\begin{table}
\renewcommand{\arraystretch}{0.75}
        \begin{tabular}{ | l | p{5cm} | p{2.5cm} | p{2cm}| p{2cm}| p{3cm} |}
 
        \hline
        Indicator & Calculation & Reference point & Indicator ratio & Expected value & Property\\ \hline
	$L_{max5$ & Mean length of largest 5\%  & $L_{\infty}$ & $Lmax5\% / L_{\infty}$ & $> 0.8$ & Conservation (large individuals) \\
        \hline
        $L_{max95\%}$ & $95^{th}$ percentile & $L_{\infty}$ & $L95\% / L_{\infty}$ & $> 0.8$ & Conservation (large individuals) \\
        \hline
	$L_{25\%}$ & $25^{th}$ percentile of length distribution & $L_{mat}$ & $L25\% / L_{mat}$ & $> 1$ & Conservation (immatures) \\
        \hline
	$L_c$ & Length at 50\% of modal abundance & $L_{mat}$ & $L_c/L_{mat}$ & $>1$ & Conservation (immatures) \\
        \hline
	$L_{mean}$ & Mean length of individuals  $> L_c$ & $L_{opt}$ = 2/3  $L_{\infty}$ & $L_{mean}/L_{opt}$ & $≈ 1$ & Optimal yield \\
        \hline
	$L_{maxY}$ & Length class with maximum biomass in catch & $L_{opt}$ = 2/3  $L_{\infty}$ &$L_{maxY} / L_{opt}$ & $=1$ & Optimal yield \\
        \hline
	$L_{mean}$ & Mean length of individuals $> L_c$ & $L_{opt}$ = 2/3  $L_{\infty}$ &$L_{mean} / L_{F=M}$ & $\geq 1$ & MSY \\
        \hline
	$L_{bar}$ & Mean length & $L_{F=M} = 0.75L_c + 0.25L_{\infty}$ & $L_{mat}$ & $> 1$ &  \\
        \hline
	$P_{mega}$ & Proportion of individuals above $L_{opt}$ + 10\%, $L_{opt}$ is estimated from $L_{\infty}$. &  & $P_{mega}$ & $> 0.3$ & Conservation (large individuals) \\
        \hline
    \end{tabular}
\end{table}    
\end{center}

\end{landscape}



\subsubsection{Receiver Operating Characteristics}

Indicators and reference point may be biased and have poor precision due to uncertainty about life history parameters, lags between size distribution and exploitation levels, variability in recruitment and resonant cohort effects that can produce long term fluctuation in the time series \citep{botsford2014cohort,bjoernstad2004trends}. This means that the reference level that can best identify the system state is unlikely to be $L_{mean}/L_{F=M}$ but a multiple of it, i.e. the discrimination threshold. The discrimination threshold is the value of $L_{mean}/L_{F=M}$ at which the a stock is said to be undergoing overfishing. A ROC curve plots the true positive rate (TPR) against the false positive rate  (FPR) at various threshold settings. Risks are asymmetric, i.e. the risk of indicating overfishing is occurring when the stock is sustainably exploited is not the same as the risk of failing to identify overfishing, and so it may be desirable to adjust the threshold to increase or decrease the sensitivity to false positives.

The ROC curve is constructed by sorting the observed outcomes from the OM by their predicted scores (from the OEM) with the highest scores first. The cumulative True Positive Rate (TPR) and True Negative Rate (TNR) are then calculated for the ordered observed outcomes. The ROC curve can also be thought of as a plot of the power as a function of the Type I Error of the decision rule. If the probability distributions for both detection and false alarm are known, the ROC curve can be generated by plotting the cumulative distribution function (area under the probability distribution from  to the discrimination threshold) of the detection probability in the y-axis versus the cumulative distribution function of the false-alarm probability on the x-axis. A ROC analysis therefore provides a tool to select the best candidate indicators. 


\section{Results}


\textbf{Figure} \ref{fig:lh} Life history parameters

\textbf{Figure} \ref{fig:om} shows the simulated exploitation history relative to $MSY$ reference points for each of the stocks for the base case. Fishing was initially low then increased to twice $F_{MSY}$, following which a recovery plan was implemented in order to reduce F to $F_{MSY}$. The coloured regions indicate exploitation levels.

\textbf{Figure} \ref{fig:lfd} Example of simulated length samples 

\textbf{Figure} \ref{fig:indicators} Example indicators with worm plots

\textbf{Figure} \ref{fig:roc} ROC curves for ICES indicators

\textbf{Figure} \ref{fig:discrim} boxplots comparing ICES indicators/reference points to ROC tresholds

\textbf{Figure} \ref{fig:tree} Regression Tree


\section{Discussion}


\begin{description}
 \item[Uncertainty]  
 \item[Risk]     
 \item[Management Frameworks] 
 \item[Robustness]
 \item[Lessons for data poor case studies]  
 \item[Lessons for data rich case studies] 
\end{description}

%For risks to be managed in a consistent way given the range of uncertainties across data rich and poor stocks requires OMs to be condition on appropriate processes. This allows an evaluation of the relative value-of-information and the value-of-control. In the former case this involves demonstrating the benefit of obtain better knowledge and data, i.e. to move a stock between categories, and in the later to consider alternative forms of indicators and control rules to develop robust advice. ROC curves are related in a direct and natural way to a cost/benefit analysis of diagnostic decision making, i.e. can be used to identify the value of control

%Setting of appropriate reference levels in the f and b component of the rules, and the extent to which this could be done with tuning that depends on life-history traits and/or the nature of the time-series was addressed using tools developed under the MyDas project. The aim of MyDas is to develop and test a range of assessment models and methods to establish Maximum Sustainable Yield (MSY), or proxy MSY reference points across the spectrum of data-limited stocks. To tune a catch rule of the form r f b, requires selecting indicators and reference points for each of the r and b components and then finding multipliers and thresholds for the component in order to combine them into a single rule. Doing this on a stock specific basis can take considerable time using MSE alone, especially as a variety of indicators and reference points can be used.  Therefore an example was developed using a Receiver Operating Characteristic or ROC curve, to show how potential indicators and reference points can be screened for a range of uncertainties before conducting MSE. 


\section{Conclusions}

\begin{itemize}
 \item 
 \item 
 \item 
 \item 
\end{itemize}

\section{Acknowledgement}

Laurence Kell's involvement was funded through the MyDas project under the Marine Biodiversity Scheme which is financed by the Irish government and the European Maritime and Fisheries Fund (EMFF) as part of the EMFF Operational Programme for 2014-2020. 

\clearpage
%\bibliography{/home/laurence/Desktop/sea++/mydas/papers/refs.bib}
%\bibliographystyle{apalike}
\documentclass[12pt,doublespacing,a4paper]{ouparticle}

%\usepackage[scaled]{helvet}
%\renewcommand\familydefault{\sfdefault} 
%\usepackage[T1]{fontenc}


\usepackage[authoryear]{natbib}
\usepackage{lineno}
\usepackage{pdfpages}
\usepackage{lscape}


\begin{document}

The use of trends in an index without a reference level (ToR iv) were explored using methods developed under the mydas project.  To do this Management Strategy Evaluation (MSE) was conducted to evaluate an empirical harvest control rule (HCR) based on a trend in an index of abundance. 

The  Operating Model (OM) was conditioning on turbot life history characteristics and the HCR was based on that used by the Commission for the Conservation of Southern Bluefin Tuna (CCSBT). The HCR has several parameters that require tuning (Hillary et al., 2016). When tuning a HCR the parameters are found by choosing values that best meet the objectives of  asset and stakeholders, i.e. optimises the outcomes modelled as a reward function. 

The HCR was modelled as part of a  Management Procedure (MP) where catches are increased when the trend in an index of abundance is positive, and decreased if the trend is negative, namely

where  is the slope in the regression of  against year for the most recent n years and  and  are  the tunable parameters andactions asymmetry so that decreases in the index do not result in the same relative change as as an increase.
When tuning an empirical MP it is run for a range of control parameters values (i.e. for,and ). These are then chosen based on the performance of the MP, i.e.  maximising a reward function based on management objectives. It can be difficult, however, to specify a single reward function, due to trade-offs between multiple objectives. Deciding which is the “best” MP therefore requires an iterative process involving managers, asset holders, stakeholders and scientists.
Once objectives are agreed the traditional way to find the control parameters is to perform  a grid search, i.e. an exhaustive search through a manually specified set of control parameters. Even for a limited number of control parameters this can take a substantial amount of computing time. Tuning was performed using random search where control parameters are selected from all the potential combinations at random.  Random search has proven to yield better results in comparison to grid search. Drawbacks of random search are that it may yields high variance during computing and since the selection of parameters is completely random no intelligence is used to sample the combinations and so lick plays its part. 

Trade offs between multiple objectives were evaluated by identifying pareto-optimal solutions (Mishra, et al., 2002) using support vector regression (SVR, Smola and Schölkopf, 2004). The best HCR parameters were then identified using a Genetic Algorithm (GA, Whitley, 1994). Both SVR and GAs are machine learning techniques. 
In optimisation studies with multi-objectives the focus is usually on finding a global optimum, i.e. the global Pareto-optimal frontier, representing the best possible objective values (Deb and Gupta, 2005).  However, in fisheries  there is usually high uncertainty about resource dynamics and solutions are therefore sensitive to the assumptions and environmental variability.  Therefore rather than finding global solution it is more important to find  robust solutions which are insensitive to uncertainty about processes. 
Figure 1 shows the trade-off between yield (Yield:MSY) and safety (the minimum expected recruitment relative to). Individual MSE (blue) results are highly variable due to variablity in recruitment and the Index of abundance used in the MP. The pareto frontier   derived from SVR are shown (red) and an example of an optimal solution highlighted (large dot).
Figures 2 show the calibration curves, obtained using the GA for the control parameters  and . This was obtained from the pareto frontiers by finding the values that corresponded to the optimal solution. If the management objectives are agreed the corresponding control value can be read off from the Y-axes. The scatter of points reflects that the Pareto frontiers are hyper-dimensional surfaces projected into 2 dimensions.
Once the control parameters that best met the management objectives were found the MSE was run for the control parameters for 2 scenarios corresponding to the Index of abundance CV (10%, 20% and 30%) and the number of years (3, 5, and 7 column) used in the regression to estimate the trend in the index; the summary statistics are shown in Figure 3.
An objective of the approach was to develop a risk based framework for conducting MSE, by allowing asset and stake holders to more easily to evaluate the trade-offs between management objectives and the impact of uncertainty when conducting MSE. The framework also provides an efficient way of tuning Management Procedures so that case specific management strategies can more easily be developed. However, since random search was used the outcomes partly depend on chance, the next step is to add intelligence by using machine learning to choose the control parameters.

The approached used demonstrates a potential stepwise procedure for conducting MSE namely
        ◦ First a single MSE is run using random search and the Pareto frontiers found.
        ◦ Objectives can be elicited from asset and stakeholders, and the trade-offs between them evaluated.
        ◦ Using the Pareto frontiers the control parameters can be derived by calibration.
        ◦ Next a set of robustness trials, can be developed for an agreed set of OMs that reflect the main uncertainties and the corresponding Pareto frontiers derived.
        ◦ A final set of control parameters can then be agreed following dialogue with asset and stakeholders




Figure 1. The trade-off between yield (Yield:MSY) and the average SSB relative to  are shown for the individual management strategy evaluations (blue) along with the pareto frontier (red).

Figure 2. Calibration regression values for the control parameters K1 and K2 for the pareto frontier for , large point is for safety~0.7.


Figure 3. Summary statistics from MSE.
References

Deb, K. and Gupta, H., 2005, March. Searching for robust Pareto-optimal solutions in multi-objective optimization. In International Conference on Evolutionary Multi-Criterion Optimization (pp. 150-164). Springer, Berlin, Heidelberg.

Hillary, R.M., Preece, A.L., Davies, C.R., Kurota, H., Sakai, O., Itoh, T., Parma, A.M., Butterworth, D.S., Ianelli, J. and Branch, T.A., 2016. A scientific alternative to moratoria for rebuilding depleted international tuna stocks. Fish and fisheries, 17(2), pp.469-482.

Smola, A.J. and Schölkopf, B., 2004. A tutorial on support vector regression. Statistics and computing, 14(3), pp.199-222.

Whitley, D., 1994. A genetic algorithm tutorial. Statistics and computing, 4(2), pp.65-85.

.

\title{An Evaluation of the Robustness of Length Based Indicators using Receiver Operating Characteristics.}

\author{%
\name{Laurence T. Kell}
\address{Centre for Environmental Policy, Imperial College London, London SW7 1NE}
\email{e-mail Lauriel@seapluslus.co.uk}
\and
\name{Co\'il\'in Minto}
\address{GMIT}
\email{Hans Gerritsen}
\address{Marine Institute}
}

\abstract{
%Scientific management advice for data rich stocks is based upon biological points and expert opinion to fix or meta-analyses to develop priors for difficult to estimate parameters. Whilst advice for data poor stocks is based upon life history parameters and outputs from data rich stock assessments to develop priors and propose indicators based upon life history traits.
%To help propose and develop robust and non-redundant productivity indicators the reliability and stability of the indices are evaluated with reference to the data rich set. 
}

\date{\today}

\keywords{}
 
\maketitle

\newpage


\linenumbers
\linespread{2}

\section{Outline}
\begin{itemize}
 \item Robustness, despite what you dont know
 \item Value-of-information
 \item Value-of-control
 \item Trends v. absolute
 \item Indicator for life histories
 \item Indicator for fisheries
\end{itemize}


\section{Introduction}

A substantial fraction of world fisheries are conducted on resources for which data are insufficient to conduct formal stock assessments and so status, productivity, and exploitation levels of many stocks and species are largely unknown \citep{thorson2015introduction}. Such stocks include by-caught, threatened, endangered, and protected species and are termed data-poor, data-limited, or information-poor. The adoption of the precautionary approach \citep[PA,][]{garcia1996precautionary}, however, still requires management plans and biological reference points for all stocks not just for the main commercial stocks where analytical assessments are available \citep{sainsbury2003ref}. 

Reference points are used in management plans as targets to maximise surplus production and as limits to minimise the risk of depleting a resource to a level where productivity is compromised. They must integrate dynamic processes such as growth, recruitment, mortality and connectivity into indices for exploitation level and spawning reproductive potential \citep{kell2015spawning}. In data poor situations life history parameters, such as maximum size and size at first maturity have been used as proxies for productivity \citep{roff1984evolution,jensen1996beverton,caddy1998short,reynolds2001life,denney2002life}. For example ICES uses indicators and proxy maximum sustainable yield (MSY) reference points based on length (e.g. the von Bertalanffy growth parameter $L_{\infty}$ and length at maturity $L_{mat}$) as part of a Precautionary Approach for stocks where only trends in mean length are available \citep[][]{ref}. 

An important characteristic of PA advice is that it should be robust to uncertainty. Where a robust control system is one that still functions correctly despite the presence of uncertainty \citep{radatz1990ieee, zhou1996robust}, while to be robust an indicator should be both reliable and stable. An indicator has high reliability if despite uncertainty it provides an accurate result, and it is stable if despite random error, similar results are produced across multiple trials. To evaluate the robustness of length based indicators and $MSY$ proxies  a simulation and sampling (i.e. sim-sam) procedure was employed where an Operatating Model (OM) was conditioned on life history parameters then psuedo data simulated using an Observation Error Model (OEM), allowing the indicators to be compared to the actual state of the resource.


\section{Material and Methods}

OMs were conditioned on life histories representing a range of population dynaics, namely turbot, brill, thornback ray, pollack and sprat. Scenarios that represent the main sources of uncertainty were also developed. Resource dynamics were simulated assuming an increasing trend in fishing mortality ($F$) that led to overfishing, then a recovery plan was implemented to bring fishing back to the $F_{MSY}$ level. Length based indicators were then generates using the OEM and compared to the fishing mortality in the OM, using Reciever Operating Characteristic (ROC) curves \citep{green1966signal}. A ROC curve can be thought of as a plot of the power as a function of the Type I Error of the decision rule. If the probability distributions for both detection and false alarm are known, the ROC curve is generated by plotting the cumulative distribution function (area under the probability distribution from  to the discrimination threshold) of the detection probability in the y-axis versus the cumulative distribution function of the false-alarm probability on the x-axis. A ROC analysis therefore provides a tool to select the best candidate indicators. 

\subsection{Methods}

For each species the life history parameters (\textbf{Table} \ref{tab:lh}) were used to parameterise a \cite{vonbert1957quantitative} growth curve, proportion mature-at-age, natural mortality \citep{lorenzen2002density} and a \cite{beverton1993dynamics} stock recruit relationship. Spawning stock biomass (SSB) was was used as a proxy for stock reproductive potential \citep[SRP][]{trippel_estimation_1999}. This assumes that fecundity is proportional to the mass-at-age of the sexually mature portion of the population irrespective of the demographic composition of adults \citep{murawski_impacts_2001} and that processes such as sexual maturity are simple functions of age \citep{matsuda_inconsistency_1996} and independent of gender.

These processes allow an equilibrium per-recruit model to be parameterised, which when combined with a stock recruitment relationship \cite{sissenwine1987alternative} is then used to condition a forward projection model to simulate the time series.

\subsubsection{Scenarios}

Even for data-rich stock assessments there is often large uncertainty about the dynamics \citep[i.e. model uncertainty;][]{punt2008refocusing}. This means that estimates of stock status are highly sensitive to assumptions about natural mortality-at-age \citep{jiao2012modelling}, vulnerability of age classes to the fisheries \citep{brooks2009analytical}, and the relationship between stock and recruitment which is difficult to estimate in practice \citep[e.g.][]{vert2013frequency,szuwalski2014examining,cury2014resolving,kell2015spawning,pepin2015reconsidering}. Therefore scenarios \citep[][]{ono2015importance,kell2015spawning,boorman1997recognising} representing these sources of uncertainty were developed for selection pattern, natural mortality, steepness of the stock recruitment relationship, recruitment variablility and sample size (\textbf{Table} \ref{tab:scen})

\begin{itemize}
 \item Selection pattern: \textbf{same as matirity ogive}; dome shaped; constant across all ages.
 \item Natural Mortality: \textbf{varies by age}; 0.2 
 \item Steepness of the stock recruitment: \textbf{0.7}; 0.9 
 \item Recruitment CV: \textbf{30\%}; \textbf{50\%}; \textbf{AR}
 \item Sample Size:\textbf{500}, \textbf{250}
\end{itemize}


\subsubsection{Indicators}


%\cite{punt2011among}


Simple catch rules have been developed for data poor stock, for example the “2 over 3” rule  of ICES (2012) which aims to keep stocks at their current level by multiplying recent catches by the trend in a biomass index. Catch rules that make use of more data sources have subsequently been developed (e.g. Fischer et al., submitted), using a rule of the form

\begin{equation} C_{y+1} = C_{y-1}rfb \end{equation}

The advised catch is based on the previous catch $C_{y-1}$, multiplied by three components $r$, $f$ and $b$, each representing a particular stock characteristic. Component $r$ corresponds to the trend in a biomass index, component $f$ uses a proxy  for the rateio $F:F_{MSY}$ based on length, and $b$ is a safeguard which protects the stock by not allowing catch to increase unlimitedly once the biomass index drops below a threshold. In this study we focus on the $f$ component, although the same approach can be used to screen any data source or even outputs from data rich assessments.

Empirical indicators based on length, used as proxies for exploitation rate, include

\begin{itemize}
 \item $L_{max5\%}$ mean length of largest 5\%
 \item $L_{95\%}$ $95^{th}$ percentile
 \item $P_{mega}$ Proportion of individuals above $L_{opt} + 10\%$
 \item $L_{25\%}$ $25^{th}$ percentile of length distribution
 \item $L_{c}$ Length at $50\%$ of modal abundance
 \item $L_{mean}$ Mean length of individuals $> L_c$
 \item $L_{max_{y}}$ Length class with maximum biomass in catch
 \item $L_{mean}$ Meanlength of individuals $> L$
\end{itemize}

and potential reference points include

\begin{itemize}
 \item $L_{\infty}$
 \item $L_{mat}$
 \item $L_{opt} = L_{\infty}\frac{3}{3+\frac{M}{K}}$, assuming $M/K = 1.5$ gives $\frac{2}{3}L_{\infty}$
 \item $L_{F=M} =  0,75l_c+0.25l_{\infty}$
\end{itemize}


\begin{landscape}

\begin{center}
\begin{table}
\renewcommand{\arraystretch}{0.75}
        \begin{tabular}{ | l | p{5cm} | p{2.5cm} | p{2cm}| p{2cm}| p{3cm} |}
 
        \hline
        Indicator & Calculation & Reference point & Indicator ratio & Expected value & Property\\ \hline
	$L_{max5$ & Mean length of largest 5\%  & $L_{\infty}$ & $Lmax5\% / L_{\infty}$ & $> 0.8$ & Conservation (large individuals) \\
        \hline
        $L_{max95\%}$ & $95^{th}$ percentile & $L_{\infty}$ & $L95\% / L_{\infty}$ & $> 0.8$ & Conservation (large individuals) \\
        \hline
	$L_{25\%}$ & $25^{th}$ percentile of length distribution & $L_{mat}$ & $L25\% / L_{mat}$ & $> 1$ & Conservation (immatures) \\
        \hline
	$L_c$ & Length at 50\% of modal abundance & $L_{mat}$ & $L_c/L_{mat}$ & $>1$ & Conservation (immatures) \\
        \hline
	$L_{mean}$ & Mean length of individuals  $> L_c$ & $L_{opt}$ = 2/3  $L_{\infty}$ & $L_{mean}/L_{opt}$ & $≈ 1$ & Optimal yield \\
        \hline
	$L_{maxY}$ & Length class with maximum biomass in catch & $L_{opt}$ = 2/3  $L_{\infty}$ &$L_{maxY} / L_{opt}$ & $=1$ & Optimal yield \\
        \hline
	$L_{mean}$ & Mean length of individuals $> L_c$ & $L_{opt}$ = 2/3  $L_{\infty}$ &$L_{mean} / L_{F=M}$ & $\geq 1$ & MSY \\
        \hline
	$L_{bar}$ & Mean length & $L_{F=M} = 0.75L_c + 0.25L_{\infty}$ & $L_{mat}$ & $> 1$ &  \\
        \hline
	$P_{mega}$ & Proportion of individuals above $L_{opt}$ + 10\%, $L_{opt}$ is estimated from $L_{\infty}$. &  & $P_{mega}$ & $> 0.3$ & Conservation (large individuals) \\
        \hline
    \end{tabular}
\end{table}    
\end{center}

\end{landscape}



\subsubsection{Receiver Operating Characteristics}

Indicators and reference point may be biased and have poor precision due to uncertainty about life history parameters, lags between size distribution and exploitation levels, variability in recruitment and resonant cohort effects that can produce long term fluctuation in the time series \citep{botsford2014cohort,bjoernstad2004trends}. This means that the reference level that can best identify the system state is unlikely to be $L_{mean}/L_{F=M}$ but a multiple of it, i.e. the discrimination threshold. The discrimination threshold is the value of $L_{mean}/L_{F=M}$ at which the a stock is said to be undergoing overfishing. A ROC curve plots the true positive rate (TPR) against the false positive rate  (FPR) at various threshold settings. Risks are asymmetric, i.e. the risk of indicating overfishing is occurring when the stock is sustainably exploited is not the same as the risk of failing to identify overfishing, and so it may be desirable to adjust the threshold to increase or decrease the sensitivity to false positives.

The ROC curve is constructed by sorting the observed outcomes from the OM by their predicted scores (from the OEM) with the highest scores first. The cumulative True Positive Rate (TPR) and True Negative Rate (TNR) are then calculated for the ordered observed outcomes. The ROC curve can also be thought of as a plot of the power as a function of the Type I Error of the decision rule. If the probability distributions for both detection and false alarm are known, the ROC curve can be generated by plotting the cumulative distribution function (area under the probability distribution from  to the discrimination threshold) of the detection probability in the y-axis versus the cumulative distribution function of the false-alarm probability on the x-axis. A ROC analysis therefore provides a tool to select the best candidate indicators. 


\section{Results}


\textbf{Figure} \ref{fig:lh} Life history parameters

\textbf{Figure} \ref{fig:om} shows the simulated exploitation history relative to $MSY$ reference points for each of the stocks for the base case. Fishing was initially low then increased to twice $F_{MSY}$, following which a recovery plan was implemented in order to reduce F to $F_{MSY}$. The coloured regions indicate exploitation levels.

\textbf{Figure} \ref{fig:lfd} Example of simulated length samples 

\textbf{Figure} \ref{fig:indicators} Example indicators with worm plots

\textbf{Figure} \ref{fig:roc} ROC curves for ICES indicators

\textbf{Figure} \ref{fig:discrim} boxplots comparing ICES indicators/reference points to ROC tresholds

\textbf{Figure} \ref{fig:tree} Regression Tree


\section{Discussion}


\begin{description}
 \item[Uncertainty]  
 \item[Risk]     
 \item[Management Frameworks] 
 \item[Robustness]
 \item[Lessons for data poor case studies]  
 \item[Lessons for data rich case studies] 
\end{description}

%For risks to be managed in a consistent way given the range of uncertainties across data rich and poor stocks requires OMs to be condition on appropriate processes. This allows an evaluation of the relative value-of-information and the value-of-control. In the former case this involves demonstrating the benefit of obtain better knowledge and data, i.e. to move a stock between categories, and in the later to consider alternative forms of indicators and control rules to develop robust advice. ROC curves are related in a direct and natural way to a cost/benefit analysis of diagnostic decision making, i.e. can be used to identify the value of control

%Setting of appropriate reference levels in the f and b component of the rules, and the extent to which this could be done with tuning that depends on life-history traits and/or the nature of the time-series was addressed using tools developed under the MyDas project. The aim of MyDas is to develop and test a range of assessment models and methods to establish Maximum Sustainable Yield (MSY), or proxy MSY reference points across the spectrum of data-limited stocks. To tune a catch rule of the form r f b, requires selecting indicators and reference points for each of the r and b components and then finding multipliers and thresholds for the component in order to combine them into a single rule. Doing this on a stock specific basis can take considerable time using MSE alone, especially as a variety of indicators and reference points can be used.  Therefore an example was developed using a Receiver Operating Characteristic or ROC curve, to show how potential indicators and reference points can be screened for a range of uncertainties before conducting MSE. 


\section{Conclusions}

\begin{itemize}
 \item 
 \item 
 \item 
 \item 
\end{itemize}

\section{Acknowledgement}

Laurence Kell's involvement was funded through the MyDas project under the Marine Biodiversity Scheme which is financed by the Irish government and the European Maritime and Fisheries Fund (EMFF) as part of the EMFF Operational Programme for 2014-2020. 

\clearpage
%\bibliography{/home/laurence/Desktop/sea++/mydas/papers/refs.bib}
%\bibliographystyle{apalike}
\documentclass[12pt,doublespacing,a4paper]{ouparticle}

%\usepackage[scaled]{helvet}
%\renewcommand\familydefault{\sfdefault} 
%\usepackage[T1]{fontenc}


\usepackage[authoryear]{natbib}
\usepackage{lineno}
\usepackage{pdfpages}
\usepackage{lscape}


\begin{document}

The use of trends in an index without a reference level (ToR iv) were explored using methods developed under the mydas project.  To do this Management Strategy Evaluation (MSE) was conducted to evaluate an empirical harvest control rule (HCR) based on a trend in an index of abundance. 

The  Operating Model (OM) was conditioning on turbot life history characteristics and the HCR was based on that used by the Commission for the Conservation of Southern Bluefin Tuna (CCSBT). The HCR has several parameters that require tuning (Hillary et al., 2016). When tuning a HCR the parameters are found by choosing values that best meet the objectives of  asset and stakeholders, i.e. optimises the outcomes modelled as a reward function. 

The HCR was modelled as part of a  Management Procedure (MP) where catches are increased when the trend in an index of abundance is positive, and decreased if the trend is negative, namely

where  is the slope in the regression of  against year for the most recent n years and  and  are  the tunable parameters andactions asymmetry so that decreases in the index do not result in the same relative change as as an increase.
When tuning an empirical MP it is run for a range of control parameters values (i.e. for,and ). These are then chosen based on the performance of the MP, i.e.  maximising a reward function based on management objectives. It can be difficult, however, to specify a single reward function, due to trade-offs between multiple objectives. Deciding which is the “best” MP therefore requires an iterative process involving managers, asset holders, stakeholders and scientists.
Once objectives are agreed the traditional way to find the control parameters is to perform  a grid search, i.e. an exhaustive search through a manually specified set of control parameters. Even for a limited number of control parameters this can take a substantial amount of computing time. Tuning was performed using random search where control parameters are selected from all the potential combinations at random.  Random search has proven to yield better results in comparison to grid search. Drawbacks of random search are that it may yields high variance during computing and since the selection of parameters is completely random no intelligence is used to sample the combinations and so lick plays its part. 

Trade offs between multiple objectives were evaluated by identifying pareto-optimal solutions (Mishra, et al., 2002) using support vector regression (SVR, Smola and Schölkopf, 2004). The best HCR parameters were then identified using a Genetic Algorithm (GA, Whitley, 1994). Both SVR and GAs are machine learning techniques. 
In optimisation studies with multi-objectives the focus is usually on finding a global optimum, i.e. the global Pareto-optimal frontier, representing the best possible objective values (Deb and Gupta, 2005).  However, in fisheries  there is usually high uncertainty about resource dynamics and solutions are therefore sensitive to the assumptions and environmental variability.  Therefore rather than finding global solution it is more important to find  robust solutions which are insensitive to uncertainty about processes. 
Figure 1 shows the trade-off between yield (Yield:MSY) and safety (the minimum expected recruitment relative to). Individual MSE (blue) results are highly variable due to variablity in recruitment and the Index of abundance used in the MP. The pareto frontier   derived from SVR are shown (red) and an example of an optimal solution highlighted (large dot).
Figures 2 show the calibration curves, obtained using the GA for the control parameters  and . This was obtained from the pareto frontiers by finding the values that corresponded to the optimal solution. If the management objectives are agreed the corresponding control value can be read off from the Y-axes. The scatter of points reflects that the Pareto frontiers are hyper-dimensional surfaces projected into 2 dimensions.
Once the control parameters that best met the management objectives were found the MSE was run for the control parameters for 2 scenarios corresponding to the Index of abundance CV (10%, 20% and 30%) and the number of years (3, 5, and 7 column) used in the regression to estimate the trend in the index; the summary statistics are shown in Figure 3.
An objective of the approach was to develop a risk based framework for conducting MSE, by allowing asset and stake holders to more easily to evaluate the trade-offs between management objectives and the impact of uncertainty when conducting MSE. The framework also provides an efficient way of tuning Management Procedures so that case specific management strategies can more easily be developed. However, since random search was used the outcomes partly depend on chance, the next step is to add intelligence by using machine learning to choose the control parameters.

The approached used demonstrates a potential stepwise procedure for conducting MSE namely
        ◦ First a single MSE is run using random search and the Pareto frontiers found.
        ◦ Objectives can be elicited from asset and stakeholders, and the trade-offs between them evaluated.
        ◦ Using the Pareto frontiers the control parameters can be derived by calibration.
        ◦ Next a set of robustness trials, can be developed for an agreed set of OMs that reflect the main uncertainties and the corresponding Pareto frontiers derived.
        ◦ A final set of control parameters can then be agreed following dialogue with asset and stakeholders




Figure 1. The trade-off between yield (Yield:MSY) and the average SSB relative to  are shown for the individual management strategy evaluations (blue) along with the pareto frontier (red).

Figure 2. Calibration regression values for the control parameters K1 and K2 for the pareto frontier for , large point is for safety~0.7.


Figure 3. Summary statistics from MSE.
References

Deb, K. and Gupta, H., 2005, March. Searching for robust Pareto-optimal solutions in multi-objective optimization. In International Conference on Evolutionary Multi-Criterion Optimization (pp. 150-164). Springer, Berlin, Heidelberg.

Hillary, R.M., Preece, A.L., Davies, C.R., Kurota, H., Sakai, O., Itoh, T., Parma, A.M., Butterworth, D.S., Ianelli, J. and Branch, T.A., 2016. A scientific alternative to moratoria for rebuilding depleted international tuna stocks. Fish and fisheries, 17(2), pp.469-482.

Smola, A.J. and Schölkopf, B., 2004. A tutorial on support vector regression. Statistics and computing, 14(3), pp.199-222.

Whitley, D., 1994. A genetic algorithm tutorial. Statistics and computing, 4(2), pp.65-85.

.

\title{An Evaluation of the Robustness of Length Based Indicators using Receiver Operating Characteristics.}

\author{%
\name{Laurence T. Kell}
\address{Centre for Environmental Policy, Imperial College London, London SW7 1NE}
\email{e-mail Lauriel@seapluslus.co.uk}
\and
\name{Co\'il\'in Minto}
\address{GMIT}
\email{Hans Gerritsen}
\address{Marine Institute}
}

\abstract{
%Scientific management advice for data rich stocks is based upon biological points and expert opinion to fix or meta-analyses to develop priors for difficult to estimate parameters. Whilst advice for data poor stocks is based upon life history parameters and outputs from data rich stock assessments to develop priors and propose indicators based upon life history traits.
%To help propose and develop robust and non-redundant productivity indicators the reliability and stability of the indices are evaluated with reference to the data rich set. 
}

\date{\today}

\keywords{}
 
\maketitle

\newpage


\linenumbers
\linespread{2}

\section{Outline}
\begin{itemize}
 \item Robustness, despite what you dont know
 \item Value-of-information
 \item Value-of-control
 \item Trends v. absolute
 \item Indicator for life histories
 \item Indicator for fisheries
\end{itemize}


\section{Introduction}

A substantial fraction of world fisheries are conducted on resources for which data are insufficient to conduct formal stock assessments and so status, productivity, and exploitation levels of many stocks and species are largely unknown \citep{thorson2015introduction}. Such stocks include by-caught, threatened, endangered, and protected species and are termed data-poor, data-limited, or information-poor. The adoption of the precautionary approach \citep[PA,][]{garcia1996precautionary}, however, still requires management plans and biological reference points for all stocks not just for the main commercial stocks where analytical assessments are available \citep{sainsbury2003ref}. 

Reference points are used in management plans as targets to maximise surplus production and as limits to minimise the risk of depleting a resource to a level where productivity is compromised. They must integrate dynamic processes such as growth, recruitment, mortality and connectivity into indices for exploitation level and spawning reproductive potential \citep{kell2015spawning}. In data poor situations life history parameters, such as maximum size and size at first maturity have been used as proxies for productivity \citep{roff1984evolution,jensen1996beverton,caddy1998short,reynolds2001life,denney2002life}. For example ICES uses indicators and proxy maximum sustainable yield (MSY) reference points based on length (e.g. the von Bertalanffy growth parameter $L_{\infty}$ and length at maturity $L_{mat}$) as part of a Precautionary Approach for stocks where only trends in mean length are available \citep[][]{ref}. 

An important characteristic of PA advice is that it should be robust to uncertainty. Where a robust control system is one that still functions correctly despite the presence of uncertainty \citep{radatz1990ieee, zhou1996robust}, while to be robust an indicator should be both reliable and stable. An indicator has high reliability if despite uncertainty it provides an accurate result, and it is stable if despite random error, similar results are produced across multiple trials. To evaluate the robustness of length based indicators and $MSY$ proxies  a simulation and sampling (i.e. sim-sam) procedure was employed where an Operatating Model (OM) was conditioned on life history parameters then psuedo data simulated using an Observation Error Model (OEM), allowing the indicators to be compared to the actual state of the resource.


\section{Material and Methods}

OMs were conditioned on life histories representing a range of population dynaics, namely turbot, brill, thornback ray, pollack and sprat. Scenarios that represent the main sources of uncertainty were also developed. Resource dynamics were simulated assuming an increasing trend in fishing mortality ($F$) that led to overfishing, then a recovery plan was implemented to bring fishing back to the $F_{MSY}$ level. Length based indicators were then generates using the OEM and compared to the fishing mortality in the OM, using Reciever Operating Characteristic (ROC) curves \citep{green1966signal}. A ROC curve can be thought of as a plot of the power as a function of the Type I Error of the decision rule. If the probability distributions for both detection and false alarm are known, the ROC curve is generated by plotting the cumulative distribution function (area under the probability distribution from  to the discrimination threshold) of the detection probability in the y-axis versus the cumulative distribution function of the false-alarm probability on the x-axis. A ROC analysis therefore provides a tool to select the best candidate indicators. 

\subsection{Methods}

For each species the life history parameters (\textbf{Table} \ref{tab:lh}) were used to parameterise a \cite{vonbert1957quantitative} growth curve, proportion mature-at-age, natural mortality \citep{lorenzen2002density} and a \cite{beverton1993dynamics} stock recruit relationship. Spawning stock biomass (SSB) was was used as a proxy for stock reproductive potential \citep[SRP][]{trippel_estimation_1999}. This assumes that fecundity is proportional to the mass-at-age of the sexually mature portion of the population irrespective of the demographic composition of adults \citep{murawski_impacts_2001} and that processes such as sexual maturity are simple functions of age \citep{matsuda_inconsistency_1996} and independent of gender.

These processes allow an equilibrium per-recruit model to be parameterised, which when combined with a stock recruitment relationship \cite{sissenwine1987alternative} is then used to condition a forward projection model to simulate the time series.

\subsubsection{Scenarios}

Even for data-rich stock assessments there is often large uncertainty about the dynamics \citep[i.e. model uncertainty;][]{punt2008refocusing}. This means that estimates of stock status are highly sensitive to assumptions about natural mortality-at-age \citep{jiao2012modelling}, vulnerability of age classes to the fisheries \citep{brooks2009analytical}, and the relationship between stock and recruitment which is difficult to estimate in practice \citep[e.g.][]{vert2013frequency,szuwalski2014examining,cury2014resolving,kell2015spawning,pepin2015reconsidering}. Therefore scenarios \citep[][]{ono2015importance,kell2015spawning,boorman1997recognising} representing these sources of uncertainty were developed for selection pattern, natural mortality, steepness of the stock recruitment relationship, recruitment variablility and sample size (\textbf{Table} \ref{tab:scen})

\begin{itemize}
 \item Selection pattern: \textbf{same as matirity ogive}; dome shaped; constant across all ages.
 \item Natural Mortality: \textbf{varies by age}; 0.2 
 \item Steepness of the stock recruitment: \textbf{0.7}; 0.9 
 \item Recruitment CV: \textbf{30\%}; \textbf{50\%}; \textbf{AR}
 \item Sample Size:\textbf{500}, \textbf{250}
\end{itemize}


\subsubsection{Indicators}


%\cite{punt2011among}


Simple catch rules have been developed for data poor stock, for example the “2 over 3” rule  of ICES (2012) which aims to keep stocks at their current level by multiplying recent catches by the trend in a biomass index. Catch rules that make use of more data sources have subsequently been developed (e.g. Fischer et al., submitted), using a rule of the form

\begin{equation} C_{y+1} = C_{y-1}rfb \end{equation}

The advised catch is based on the previous catch $C_{y-1}$, multiplied by three components $r$, $f$ and $b$, each representing a particular stock characteristic. Component $r$ corresponds to the trend in a biomass index, component $f$ uses a proxy  for the rateio $F:F_{MSY}$ based on length, and $b$ is a safeguard which protects the stock by not allowing catch to increase unlimitedly once the biomass index drops below a threshold. In this study we focus on the $f$ component, although the same approach can be used to screen any data source or even outputs from data rich assessments.

Empirical indicators based on length, used as proxies for exploitation rate, include

\begin{itemize}
 \item $L_{max5\%}$ mean length of largest 5\%
 \item $L_{95\%}$ $95^{th}$ percentile
 \item $P_{mega}$ Proportion of individuals above $L_{opt} + 10\%$
 \item $L_{25\%}$ $25^{th}$ percentile of length distribution
 \item $L_{c}$ Length at $50\%$ of modal abundance
 \item $L_{mean}$ Mean length of individuals $> L_c$
 \item $L_{max_{y}}$ Length class with maximum biomass in catch
 \item $L_{mean}$ Meanlength of individuals $> L$
\end{itemize}

and potential reference points include

\begin{itemize}
 \item $L_{\infty}$
 \item $L_{mat}$
 \item $L_{opt} = L_{\infty}\frac{3}{3+\frac{M}{K}}$, assuming $M/K = 1.5$ gives $\frac{2}{3}L_{\infty}$
 \item $L_{F=M} =  0,75l_c+0.25l_{\infty}$
\end{itemize}


\begin{landscape}

\begin{center}
\begin{table}
\renewcommand{\arraystretch}{0.75}
        \begin{tabular}{ | l | p{5cm} | p{2.5cm} | p{2cm}| p{2cm}| p{3cm} |}
 
        \hline
        Indicator & Calculation & Reference point & Indicator ratio & Expected value & Property\\ \hline
	$L_{max5$ & Mean length of largest 5\%  & $L_{\infty}$ & $Lmax5\% / L_{\infty}$ & $> 0.8$ & Conservation (large individuals) \\
        \hline
        $L_{max95\%}$ & $95^{th}$ percentile & $L_{\infty}$ & $L95\% / L_{\infty}$ & $> 0.8$ & Conservation (large individuals) \\
        \hline
	$L_{25\%}$ & $25^{th}$ percentile of length distribution & $L_{mat}$ & $L25\% / L_{mat}$ & $> 1$ & Conservation (immatures) \\
        \hline
	$L_c$ & Length at 50\% of modal abundance & $L_{mat}$ & $L_c/L_{mat}$ & $>1$ & Conservation (immatures) \\
        \hline
	$L_{mean}$ & Mean length of individuals  $> L_c$ & $L_{opt}$ = 2/3  $L_{\infty}$ & $L_{mean}/L_{opt}$ & $≈ 1$ & Optimal yield \\
        \hline
	$L_{maxY}$ & Length class with maximum biomass in catch & $L_{opt}$ = 2/3  $L_{\infty}$ &$L_{maxY} / L_{opt}$ & $=1$ & Optimal yield \\
        \hline
	$L_{mean}$ & Mean length of individuals $> L_c$ & $L_{opt}$ = 2/3  $L_{\infty}$ &$L_{mean} / L_{F=M}$ & $\geq 1$ & MSY \\
        \hline
	$L_{bar}$ & Mean length & $L_{F=M} = 0.75L_c + 0.25L_{\infty}$ & $L_{mat}$ & $> 1$ &  \\
        \hline
	$P_{mega}$ & Proportion of individuals above $L_{opt}$ + 10\%, $L_{opt}$ is estimated from $L_{\infty}$. &  & $P_{mega}$ & $> 0.3$ & Conservation (large individuals) \\
        \hline
    \end{tabular}
\end{table}    
\end{center}

\end{landscape}



\subsubsection{Receiver Operating Characteristics}

Indicators and reference point may be biased and have poor precision due to uncertainty about life history parameters, lags between size distribution and exploitation levels, variability in recruitment and resonant cohort effects that can produce long term fluctuation in the time series \citep{botsford2014cohort,bjoernstad2004trends}. This means that the reference level that can best identify the system state is unlikely to be $L_{mean}/L_{F=M}$ but a multiple of it, i.e. the discrimination threshold. The discrimination threshold is the value of $L_{mean}/L_{F=M}$ at which the a stock is said to be undergoing overfishing. A ROC curve plots the true positive rate (TPR) against the false positive rate  (FPR) at various threshold settings. Risks are asymmetric, i.e. the risk of indicating overfishing is occurring when the stock is sustainably exploited is not the same as the risk of failing to identify overfishing, and so it may be desirable to adjust the threshold to increase or decrease the sensitivity to false positives.

The ROC curve is constructed by sorting the observed outcomes from the OM by their predicted scores (from the OEM) with the highest scores first. The cumulative True Positive Rate (TPR) and True Negative Rate (TNR) are then calculated for the ordered observed outcomes. The ROC curve can also be thought of as a plot of the power as a function of the Type I Error of the decision rule. If the probability distributions for both detection and false alarm are known, the ROC curve can be generated by plotting the cumulative distribution function (area under the probability distribution from  to the discrimination threshold) of the detection probability in the y-axis versus the cumulative distribution function of the false-alarm probability on the x-axis. A ROC analysis therefore provides a tool to select the best candidate indicators. 


\section{Results}


\textbf{Figure} \ref{fig:lh} Life history parameters

\textbf{Figure} \ref{fig:om} shows the simulated exploitation history relative to $MSY$ reference points for each of the stocks for the base case. Fishing was initially low then increased to twice $F_{MSY}$, following which a recovery plan was implemented in order to reduce F to $F_{MSY}$. The coloured regions indicate exploitation levels.

\textbf{Figure} \ref{fig:lfd} Example of simulated length samples 

\textbf{Figure} \ref{fig:indicators} Example indicators with worm plots

\textbf{Figure} \ref{fig:roc} ROC curves for ICES indicators

\textbf{Figure} \ref{fig:discrim} boxplots comparing ICES indicators/reference points to ROC tresholds

\textbf{Figure} \ref{fig:tree} Regression Tree


\section{Discussion}


\begin{description}
 \item[Uncertainty]  
 \item[Risk]     
 \item[Management Frameworks] 
 \item[Robustness]
 \item[Lessons for data poor case studies]  
 \item[Lessons for data rich case studies] 
\end{description}

%For risks to be managed in a consistent way given the range of uncertainties across data rich and poor stocks requires OMs to be condition on appropriate processes. This allows an evaluation of the relative value-of-information and the value-of-control. In the former case this involves demonstrating the benefit of obtain better knowledge and data, i.e. to move a stock between categories, and in the later to consider alternative forms of indicators and control rules to develop robust advice. ROC curves are related in a direct and natural way to a cost/benefit analysis of diagnostic decision making, i.e. can be used to identify the value of control

%Setting of appropriate reference levels in the f and b component of the rules, and the extent to which this could be done with tuning that depends on life-history traits and/or the nature of the time-series was addressed using tools developed under the MyDas project. The aim of MyDas is to develop and test a range of assessment models and methods to establish Maximum Sustainable Yield (MSY), or proxy MSY reference points across the spectrum of data-limited stocks. To tune a catch rule of the form r f b, requires selecting indicators and reference points for each of the r and b components and then finding multipliers and thresholds for the component in order to combine them into a single rule. Doing this on a stock specific basis can take considerable time using MSE alone, especially as a variety of indicators and reference points can be used.  Therefore an example was developed using a Receiver Operating Characteristic or ROC curve, to show how potential indicators and reference points can be screened for a range of uncertainties before conducting MSE. 


\section{Conclusions}

\begin{itemize}
 \item 
 \item 
 \item 
 \item 
\end{itemize}

\section{Acknowledgement}

Laurence Kell's involvement was funded through the MyDas project under the Marine Biodiversity Scheme which is financed by the Irish government and the European Maritime and Fisheries Fund (EMFF) as part of the EMFF Operational Programme for 2014-2020. 

\clearpage
%\bibliography{/home/laurence/Desktop/sea++/mydas/papers/refs.bib}
%\bibliographystyle{apalike}
\input{roc.bbl}

\clearpage
\section{Figures}

\newpage
\begin{figure}[h]
\centering
%\includepdf[pages=1,pagecommand={},width=\textwidth]{roc-cor-1.pdf}
\includegraphics[width=\textwidth]{roc-cor-1.png}
\caption{Life history parameters.}
\label{fig:cor}
\end{figure}

\newpage
\begin{figure}[h]
\centering
\includegraphics[width=\textwidth]{roc-ts-1.png}
\caption{Operating Models, showing exploitation history relative to $MSY$ reference points; fishing was initially low then increased to twice $F_{MSY}$ following which a recovery plan was implemented in order to reduce $F$ to $F_{MSY}$.}
\label{fig:oms}
\end{figure}

% js!K*o8%Lo$n


\newpage
\begin{figure}[h]
\centering
\includegraphics[width=\textwidth]{roc-lf-1.png}
\caption{Simulated length samples.}
\label{fig:samples}
\end{figure}

\newpage
\begin{figure}[h]
\centering
\includegraphics[width=\textwidth]{roc-inds-1.png}
\caption{Examples of indicators for the turbot base case; the coloured regions indicate exploitation levels relative to $F_{MSY}$, $F<F_{MSY}$ (green),  $F_{MSY} <F< 1.5F_{MSY}$ (yellow), and $F \geq 1.5F_{MSY}$ (red).}
\label{fig:indicators}
\end{figure}


\newpage
\begin{figure}[h]
\centering
\includegraphics[width=\textwidth]{roc-overfish-roc-pollack-1.png}
\caption{Receiver Operating Characteristic curves.}
\label{fig:roc}
\end{figure}


\newpage
\begin{figure}[h]
\centering
\includegraphics[width=\textwidth]{roc-overfish-ref-pollack-1.png}
\caption{Descrimnation thresholds.}
\label{fig:discrim}
\end{figure}

\clearpage
\section{Appendix}


2012a. ICES Implementation of Advice for Data-limited Stocks in 2012 in its 2012 Advice. ICES CM 2012/ACOM 68: 42 pp.

ICES. 2017a. Report of the ICES Workshop on the Development of Quantitative Assessment Methodologies based on Life-history traits, exploitation characteristics, and other relevant parameters for data-limited stocks in categories 3-6 (WKLIFE VI), 3-7 October 2016, Lisb. ICES CM 2016/ACOM:59: 106 pp.


\end{document}




\clearpage
\section{Figures}

\newpage
\begin{figure}[h]
\centering
%\includepdf[pages=1,pagecommand={},width=\textwidth]{roc-cor-1.pdf}
\includegraphics[width=\textwidth]{roc-cor-1.png}
\caption{Life history parameters.}
\label{fig:cor}
\end{figure}

\newpage
\begin{figure}[h]
\centering
\includegraphics[width=\textwidth]{roc-ts-1.png}
\caption{Operating Models, showing exploitation history relative to $MSY$ reference points; fishing was initially low then increased to twice $F_{MSY}$ following which a recovery plan was implemented in order to reduce $F$ to $F_{MSY}$.}
\label{fig:oms}
\end{figure}

% js!K*o8%Lo$n


\newpage
\begin{figure}[h]
\centering
\includegraphics[width=\textwidth]{roc-lf-1.png}
\caption{Simulated length samples.}
\label{fig:samples}
\end{figure}

\newpage
\begin{figure}[h]
\centering
\includegraphics[width=\textwidth]{roc-inds-1.png}
\caption{Examples of indicators for the turbot base case; the coloured regions indicate exploitation levels relative to $F_{MSY}$, $F<F_{MSY}$ (green),  $F_{MSY} <F< 1.5F_{MSY}$ (yellow), and $F \geq 1.5F_{MSY}$ (red).}
\label{fig:indicators}
\end{figure}


\newpage
\begin{figure}[h]
\centering
\includegraphics[width=\textwidth]{roc-overfish-roc-pollack-1.png}
\caption{Receiver Operating Characteristic curves.}
\label{fig:roc}
\end{figure}


\newpage
\begin{figure}[h]
\centering
\includegraphics[width=\textwidth]{roc-overfish-ref-pollack-1.png}
\caption{Descrimnation thresholds.}
\label{fig:discrim}
\end{figure}

\clearpage
\section{Appendix}


2012a. ICES Implementation of Advice for Data-limited Stocks in 2012 in its 2012 Advice. ICES CM 2012/ACOM 68: 42 pp.

ICES. 2017a. Report of the ICES Workshop on the Development of Quantitative Assessment Methodologies based on Life-history traits, exploitation characteristics, and other relevant parameters for data-limited stocks in categories 3-6 (WKLIFE VI), 3-7 October 2016, Lisb. ICES CM 2016/ACOM:59: 106 pp.


\end{document}




\clearpage
\section{Figures}

\newpage
\begin{figure}[h]
\centering
%\includepdf[pages=1,pagecommand={},width=\textwidth]{roc-cor-1.pdf}
\includegraphics[width=\textwidth]{roc-cor-1.png}
\caption{Life history parameters.}
\label{fig:cor}
\end{figure}

\newpage
\begin{figure}[h]
\centering
\includegraphics[width=\textwidth]{roc-ts-1.png}
\caption{Operating Models, showing exploitation history relative to $MSY$ reference points; fishing was initially low then increased to twice $F_{MSY}$ following which a recovery plan was implemented in order to reduce $F$ to $F_{MSY}$.}
\label{fig:oms}
\end{figure}

% js!K*o8%Lo$n


\newpage
\begin{figure}[h]
\centering
\includegraphics[width=\textwidth]{roc-lf-1.png}
\caption{Simulated length samples.}
\label{fig:samples}
\end{figure}

\newpage
\begin{figure}[h]
\centering
\includegraphics[width=\textwidth]{roc-inds-1.png}
\caption{Examples of indicators for the turbot base case; the coloured regions indicate exploitation levels relative to $F_{MSY}$, $F<F_{MSY}$ (green),  $F_{MSY} <F< 1.5F_{MSY}$ (yellow), and $F \geq 1.5F_{MSY}$ (red).}
\label{fig:indicators}
\end{figure}


\newpage
\begin{figure}[h]
\centering
\includegraphics[width=\textwidth]{roc-overfish-roc-pollack-1.png}
\caption{Receiver Operating Characteristic curves.}
\label{fig:roc}
\end{figure}


\newpage
\begin{figure}[h]
\centering
\includegraphics[width=\textwidth]{roc-overfish-ref-pollack-1.png}
\caption{Descrimnation thresholds.}
\label{fig:discrim}
\end{figure}

\clearpage
\section{Appendix}


2012a. ICES Implementation of Advice for Data-limited Stocks in 2012 in its 2012 Advice. ICES CM 2012/ACOM 68: 42 pp.

ICES. 2017a. Report of the ICES Workshop on the Development of Quantitative Assessment Methodologies based on Life-history traits, exploitation characteristics, and other relevant parameters for data-limited stocks in categories 3-6 (WKLIFE VI), 3-7 October 2016, Lisb. ICES CM 2016/ACOM:59: 106 pp.


\end{document}




\clearpage
\section{Figures}

\newpage
\begin{figure}[h]
\centering
%\includepdf[pages=1,pagecommand={},width=\textwidth]{roc-cor-1.pdf}
\includegraphics[width=\textwidth]{roc-cor-1.png}
\caption{Life history parameters.}
\label{fig:cor}
\end{figure}

\newpage
\begin{figure}[h]
\centering
\includegraphics[width=\textwidth]{roc-ts-1.png}
\caption{Operating Models, showing exploitation history relative to $MSY$ reference points; fishing was initially low then increased to twice $F_{MSY}$ following which a recovery plan was implemented in order to reduce $F$ to $F_{MSY}$.}
\label{fig:oms}
\end{figure}

% js!K*o8%Lo$n


\newpage
\begin{figure}[h]
\centering
\includegraphics[width=\textwidth]{roc-lf-1.png}
\caption{Simulated length samples.}
\label{fig:samples}
\end{figure}

\newpage
\begin{figure}[h]
\centering
\includegraphics[width=\textwidth]{roc-inds-1.png}
\caption{Examples of indicators for the turbot base case; the coloured regions indicate exploitation levels relative to $F_{MSY}$, $F<F_{MSY}$ (green),  $F_{MSY} <F< 1.5F_{MSY}$ (yellow), and $F \geq 1.5F_{MSY}$ (red).}
\label{fig:indicators}
\end{figure}


\newpage
\begin{figure}[h]
\centering
\includegraphics[width=\textwidth]{roc-overfish-roc-pollack-1.png}
\caption{Receiver Operating Characteristic curves.}
\label{fig:roc}
\end{figure}


\newpage
\begin{figure}[h]
\centering
\includegraphics[width=\textwidth]{roc-overfish-ref-pollack-1.png}
\caption{Descrimnation thresholds.}
\label{fig:discrim}
\end{figure}

\clearpage
\section{Appendix}


2012a. ICES Implementation of Advice for Data-limited Stocks in 2012 in its 2012 Advice. ICES CM 2012/ACOM 68: 42 pp.

ICES. 2017a. Report of the ICES Workshop on the Development of Quantitative Assessment Methodologies based on Life-history traits, exploitation characteristics, and other relevant parameters for data-limited stocks in categories 3-6 (WKLIFE VI), 3-7 October 2016, Lisb. ICES CM 2016/ACOM:59: 106 pp.


\end{document}


