Case study stocks were identified based on their economic and ecological importance and then appropriate datasets were collated. Candidate assessment methods were then reviewed and the performance of alternative reference points compared. 

The common framework was developed in R, and existing assessment methods integrated with \href{http://www.flr-project.org/}{FLR}, a collection of tools for quantitative fisheries science, developed in the R language \citep{kell2007flr}. This required extending and updating existing FLR packages. For example \href{https://github.com/flr/flife}{FLife}
was then extended to include methods to build simulation models based on life history theory. A new package \href{https://github.com/flr/mydas/wiki}{mydas} was also developed to provide the tools for the common framework allowing bespoke applications to be developed.
 
Although a prototype shiny-app was produced, it was agreed that \href{https://3o2y9wugzp1kfxr5hvzgzq-on.drv.tw/MyDas/doc/html/mydas_vignettes.html}{vignettes} were a better tool since these provide examples that can be adapted and extended by others,and provide reproducible examples for developing case study applications.

Data sets for fisheries dependent and independent data and life histories were collated, and methods developed to link these to R using a variety of tools.

The work of MyDas was presented at WKLIFEVIII, WKLIFEIX, and WGMSE3 and a workshop was held at the Marine Institute. The tools developed under MyDas were used to develop a generic catch rule for data limited stocks.

Several peer review papers resulted from the project;  one paper is published, two are have been submitted and are in peer review, and three other papers are currently being finalised.

