
The goal of MyDas was to develop and test a range of assessment methods to establish Maximum Sustainable Yield (MSY) or proxy MSY reference points, across the spectrum of data-limited stocks. 

Case studies were identified based on their economic and ecological importance and then appropriate  datasets on life histories, commercial catches and surveys collated. Candidate data poor assessment methods were identified and their performance evaluated using simulation. A Management Strategy Evaluation (MSE) framework for data poor stocks was developed in R, using \href{http://www.flr-project.org/}{FLR}, a collection of tools for quantitative fisheries science, developed in the R language \citep{kell2007flr}. This required extending existing FLR packages; for example \href{https://github.com/flr/flife}{FLife} was extended to include methods to build simulation models based on life history theory. A new package \href{https://github.com/flr/mydas/wiki}{mydas} was developed to allow the assessment methods to be run with FLR and to provide tools for simulation of data advice frameworks.
 
%Although a prototype shiny-app was initially produced, it was agreed that \href{https://3o2y9wugzp1kfxr5hvzgzq-on.drv.tw/MyDas/doc/html/mydas_vignettes.html}{vignettes} were a better tool as these provide examples that can be adapted and extended by others,and provide reproducible examples for developing case study applications.

As well as liason with the Marine Institute the project supported participation at the ICES Workshop On The Development Of Quantitative Assessment Methodologies Based On Life-History Traits, Exploitation Characteristics, And Other Relevant Parameters For Data-Limited Stocks (\href{https://www.ices.dk/community/groups/Pages/WKLIFEIX.aspx}{WKLIFE}).

%The work of MyDas was presented at WKLIFEVIII, WKLIFEIX, and WGMSE3 and a workshop was held at the Marine Institute. The tools developed under MyDas were used to develop a generic catch rule for data limited stocks.

%Several peer review papers resulted from the project;  one paper is published, two are have been submitted and are in peer review, and three other papers are currently being finalised.

Outputs from MyDas included

\begin{itemize}[noitemsep,topsep=0pt,parsep=0pt,partopsep=0pt]
 \item R packages with assessment models for data-limited stocks implemented in an MSE framework. 
 \item R Vignettes describing the methods and findings.
 \item Publication in peer-reviewed journals on data poor approaches methods.
\end{itemize}

