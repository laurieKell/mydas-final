Attendance at \href{http://ices.dk/sites/pub/Publication Reports/Expert Group Report/Fisheries Resources Steering Group/2019/WKLIFEIX/WKLIFE_IX_2019.pdf#page=30}{wklifeix: 4 Length-based approaches} and 
ICES WKLIFE \& Working Group for the Celtic Seas Ecoregion \href{https://www.ices.dk/community/groups/Pages/WGCSE.aspx}{(WGCSE)}


\textit{Linkage with other projects](https://github.com/laurieKell/mydas/wiki/7-Linkage-with-other-projects) The service provider is required to link research output to the following projects: }

The MyDas framework was developed though case studies in collaboration with a range of partners, as well as the MI and ICES, these included the tuna Regional Management Fisheries Organisations (tRFMOs), the University of Washington and the JRC. These case studies resulted in a number of peer review manuscripts. 

\begin{itemize}
 \item Monkfish\\
 This project will develop in close collaboration with the Cullen Fellowship of Mr Luke Batts, co-supervised by Dr Hans Gerritsen (MI) and Dr C\'oil\'in Minto (GMIT). Active collaboration will occur with Tasks 3--5, as these are similarly proposed in the Cullen Fellowship where they are applied specifically to \textit{Lophius budegassa} and \textit{Lophius piscatorius} stocks in ICES areas VII-VIII.
 \item Pollock \\
 Active collaboration exists between GMIT and the Newport Research Cluster (e.g., \emph{Unlocking the Archive} project). Further collaboration and linkages will be built around data-poor assessment of pollock (liaising with the dedicated Scientific and Technical Officer working on pollock at the Furnace research facility). Both visits to Newport and group attendance at the monthly meetings will facilitate collaboration and crossover.
 \item DRuMFISH project \\
 The project will also link with the DGMARE project: ``Study on approaches to management for data-poor stocks in mixed fisheries (DRuMFISH)'' to which GMIT is a partner in the consortium. Methodological development from DRuMFISH (e.g., hierarchical methods) will be directly relevant to the present proposal. 
 \item CPV codes 71354500-9 Marine survey services 73112000-0 Marine research services 90712300-4 Marine conservation strategy planning 98360000-4 Marine services 77700000-7 Services incidental to fishing 73000000-2 Research and development services and related consultancy services 73110000-6 Research services 73200000-4 Research and development consultancy services 73210000-7 Research consultancy services.
 \item There are also other projects worldwide that can be link to e.g. the global group on stock assessment methods and the tRFMO MSE WG. 
 \end{itemize}

\begin{itemize}[labelindent=\parindent,noitemsep,topsep=0pt,parsep=0pt,partopsep=0pt]
 \item \textbf{Achievements}
 \item \textbf{Promises \& Differences}
 \item \textbf{What \& Why}
\end{itemize}

\href{http://ices.dk/sites/pub/Publication Reports/Expert Group Report/Fisheries Resources Steering Group/2019/WKLIFEIX/WKLIFE_IX_2019.pdf#page=30}{wklifeix: 4 Length-based approaches}

\href{http://ices.dk/sites/pub/Publication Reports/Expert Group Report/Fisheries Resources Steering Group/2019/WKLIFEIX/WKLIFE_IX_2019.pdf#page=58)}{wklifeix: ROC}

\href{http://ices.dk/sites/pub/Publication Reports/Expert Group Report/Fisheries Resources Steering Group/2019/WKLIFEIX/WKLIFE_IX_2019.pdf#page=100}{wklifeix: hake}

\href{http://ices.dk/sites/pub/Publication Reports/Expert Group Report/Fisheries Resources Steering Group/2019/WKLIFEIX/WKLIFE_IX_2019.pdf#page=101}{wklifeix: hake}

Evaluate further improvements to the performance of the WKMSYCat34 catch rule 3.2.1.
Focus on improving the catch rule for stocks with von Bertalanffy growth parameter
k>0.32, investigate more extensively the definition of the catch rule components and their
impact on performance, and investigate the possibility of alternative catch rules.
• Explore the operating model set-up for data-limited simulations, including sensitivity
analyses based on the Jacobian; e.g. elasticity analysis, on how the different life-history
and fishery parameters affect the simulated stock behaviour under exploitation, an analysis of the nature of time-series and trends of observable stock characteristics (such as
fishery-dependent and -independent metrics) and how the knowledge gained can be
used to further improve the performance of catch rules

  \*\*The International Council of the Exploration of the Sea (ICES)\*\* is in the process of developing methods to identify MSY proxy reference points for data-limited stocks (\href{http://ices.dk/sites/pub/Publication Reports/Expert Group Report/Fisheries Resources Steering Group/2019/WKLIFEIX/WKLIFE_IX_2019.pdf}{WKLIFE}). The service provider is required to contribute to this process by proposing and testing new assessment models and methods of establishing reference points and will be expected to attend up to 4 one-week meetings at ICES headquarters in Copenhagen. However there are key differences with the ICES approach. Since this research contract will include stocks not currently assessed by ICES; focusing on the available data for each stock first and on the methods second; the ICES approach focuses on the methods first and then applies a limited number of methods to a large number of stocks.
