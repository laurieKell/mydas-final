At the start of the project a number of research projects with potential links to MyDas were identified, and where possible links would be made. The projects included

\begin{itemize}
 \item Monkfish\\
 This project will develop in close collaboration with the Cullen Fellowship of Mr Luke Batts, co-supervised by Dr Hans Gerritsen (MI) and Dr C\'oil\'in Minto (GMIT). Active collaboration will occur with Tasks 3--5, as these are similarly proposed in the Cullen Fellowship where they are applied specifically to \textit{Lophius budegassa} and \textit{Lophius piscatorius} stocks in ICES areas VII-VIII.
 \item Pollock \\
 Active collaboration exists between GMIT and the Newport Research Cluster (e.g., \emph{Unlocking the Archive} project). Further collaboration and linkages will be built around data-poor assessment of pollock (liaising with the dedicated Scientific and Technical Officer working on pollock at the Furnace research facility). Both visits to Newport and group attendance at the monthly meetings will facilitate collaboration and crossover.
 \item DRuMFISH project \\
 The project will also link with the DGMARE project: ``Study on approaches to management for data-poor stocks in mixed fisheries (DRuMFISH)'' to which GMIT is a partner in the consortium. Methodological development from DRuMFISH (e.g., hierarchical methods) will be directly relevant to the present proposal. 
 \item CPV codes 71354500-9 Marine survey services 73112000-0 Marine research services 90712300-4 Marine conservation strategy planning 98360000-4 Marine services 77700000-7 Services incidental to fishing 73000000-2 Research and development services and related consultancy services 73110000-6 Research services 73200000-4 Research and development consultancy services 73210000-7 Research consultancy services.
 \item There are also other projects worldwide that can be link to e.g. the global group on stock assessment methods and the tRFMO MSE WG. 
 \end{itemize}
 
In addition the contractors attended \href{http://ices.dk/sites/pub/Publication Reports/Expert Group Report/Fisheries Resources Steering Group/2019/WKLIFEIX/WKLIFE_IX_2019.pdf#page=30}{WKLIFE} and 
the Working Group for the Celtic Seas Ecoregion \href{https://www.ices.dk/community/groups/Pages/WGCSE.aspx}{(WGCSE)} and participated in the stock assessment work of the MI through ICES. Furthermore the MyDas framework was developed though case studies in collaboration with a range of partners, as well as the MI and ICES, these included the tuna Regional Management Fisheries Organisations (tRFMOs), the University of Washington and the JRC. These case studies resulted in a number of peer review manuscripts. 

\coilin{Work is ongoing on monkfish in collaboration with The Cullen Fellowship of Mr Luke Batts, co-supervised by Dr Hans Gerritsen (MI) and Dr C\'oil\'in Minto (GMIT) has actively collaborated on Tasks 3, 4 and 5. While the MyDas framework is being used to look at potential indicators and mangement plans for Pollack.}



\begin{comment}
 
\end{comment}

%\href{http://ices.dk/sites/pub/Publication Reports/Expert Group Report/Fisheries Resources Steering Group/2019/WKLIFEIX/WKLIFE_IX_2019.pdf#page=30}{wklifeix: 4 Length-based approaches}
%\href{http://ices.dk/sites/pub/Publication Reports/Expert Group Report/Fisheries Resources Steering Group/2019/WKLIFEIX/WKLIFE_IX_2019.pdf#page=58)}{wklifeix: ROC}
%\href{http://ices.dk/sites/pub/Publication Reports/Expert Group Report/Fisheries Resources Steering Group/2019/WKLIFEIX/WKLIFE_IX_2019.pdf#page=100}{wklifeix: hake}
%\href{http://ices.dk/sites/pub/Publication Reports/Expert Group Report/Fisheries Resources Steering Group/2019/WKLIFEIX/WKLIFE_IX_2019.pdf#page=101}{wklifeix: hake}
