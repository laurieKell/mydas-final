The project relied on existing datasets rather than collecting new data and collating these data ran in parallel with other tasks. As well as Marine Institute data datasets were also obtained from ICES\footnote{\url{https://www.ices.dk/marine-data/dataset-collections/Pages/default.aspx}} and the JRC\footnote{\url{https://stecf.jrc.ec.europa.eu/dd/medbs/ram}}. While life history parameters were obtained from Fishbase\footnote{\url{http://www.fishbase.org/search.php}} and fishnets\footnote{\url{https://github.com/fishnets/fishnets}}, and stock assessment inputs and outputs from stock.assesment.org\footnote{\url{https://www.stockassessment.org}}.

A database was designed see the \hyperref[appendix:db]{database summary} in the Appendix. These data were \href{https://3o2y9wugzp1kfxr5hvzgzq-on.drv.tw/MyDas/tasks/1/stockprioritisation.nb.html}{summarised} to help identify the case studies. To select the case studies a key dataset were the life history parameters. For data poor stocks where infomation on key processes is lacking life history theory can be used allowing simulation models to be developed. since many studies have shown that relationships between life history traits exists for processes such as growth, maturity and natural mortality. See the vignette on \href{https://3o2y9wugzp1kfxr5hvzgzq-on.drv.tw/MyDas/vignettes/conditioning.html}{conditioning Operating Models} on life history parameters. 

The various preliminary analyses based are archived on \href{https://drive.google.com/drive/folders/1pzXh8j-Y4dtJikFqP7RUuwBu0XY0a9ao?usp=sharing}{google drive} 


