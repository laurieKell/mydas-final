During MyDas a number of research projects with potential links to MyDas were identified and links made. The projects included:

\begin{itemize}
 \item Pollock \\
 Active collaboration occurred between MyDas and the Newport Research Cluster, including linkages built around data-poor assessment of \href{https://3o2y9wugzp1kfxr5hvzgzq-on.drv.tw/MyDas/tasks/2/pollack.html}{pollack} (liaising with the dedicated Scientific and Technical Officer working on pollock at the Furnace research facility). The MyDas framework is being used to look at potential indicators and mangement plans for Pollack.
 \item Monkfish\\
 Direct links to Cullen fellowship project \emph{Monkfish management strategy evaluation} (CF/16/03) were established. In particular, Ph.D. candidate Luke Batts is testing stage-based stock assessment methods using monkfish stocks simulated with \href{https://github.com/flr/flife}{FLife} and linking to MyDas Tasks 3, 4 and 5. This project has made extensive use of the MSE frameworks established within MyDas and FLR to test key uncertainties in advice for monkfish stocks (specifically \textit{Lophius budegassa} and \textit{Lophius piscatorius} stocks in ICES areas VII-VIII.). 
  \item Attendance at Workshop on the Development of Quantitative Assessment Methodologies based on LIFE-history traits, exploitation characteristics, and other relevant parameters for data-limited stocks \href{https://www.ices.dk/community/groups/Pages/WKLIFEIX.aspx}{(WKLIFE)} and the Working Group for the Celtic Seas Ecoregion \href{https://www.ices.dk/community/groups/Pages/WGCSE.aspx}{(WGCSE)}, see presentations.
  \begin{itemize}
   \item \href{https://3o2y9wugzp1kfxr5hvzgzq-on.drv.tw/MyDas/presentations/mydas-wklifeix.html}{WKLIFE VIII}
   \item \href{https://3o2y9wugzp1kfxr5hvzgzq-on.drv.tw/MyDas/presentations/mydas-wklifeviii.html}{WKLIFE IX} 
 \end{itemize}
 \item Attendance at the \href{https://www.marine.ie/Home/site-area/news-events/news/first-environmetrics-forum-data-experts-ireland}{first Irish Environmetrics Forum} where methods for policy-relevant advice were presented and discussed.
 \item Furthermore the MyDas framework was developed though case studies in collaboration with a range of partners, as well as the MI and ICES, these included the tuna Regional Management Fisheries Organisations (tRFMOs), the University of Washington and the JRC. These case studies resulted in a number of peer review manuscripts.
 \end{itemize} 


\begin{comment}
 
\end{comment}

%\href{http://ices.dk/sites/pub/Publication Reports/Expert Group Report/Fisheries Resources Steering Group/2019/WKLIFEIX/WKLIFE_IX_2019.pdf#page=30}{wklifeix: 4 Length-based approaches}
%\href{http://ices.dk/sites/pub/Publication Reports/Expert Group Report/Fisheries Resources Steering Group/2019/WKLIFEIX/WKLIFE_IX_2019.pdf#page=58)}{wklifeix: ROC}
%\href{http://ices.dk/sites/pub/Publication Reports/Expert Group Report/Fisheries Resources Steering Group/2019/WKLIFEIX/WKLIFE_IX_2019.pdf#page=100}{wklifeix: hake}
%\href{http://ices.dk/sites/pub/Publication Reports/Expert Group Report/Fisheries Resources Steering Group/2019/WKLIFEIX/WKLIFE_IX_2019.pdf#page=101}{wklifeix: hake}
