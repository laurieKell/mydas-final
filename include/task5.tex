\textit{\textbf{Reference point comparisons (across candidate methods) Once reference points have been identified, their performance should be evaluated through simple management strategy evaluations.}}


\href{http://ices.dk/sites/pub/Publication Reports/Expert Group Report/Fisheries Resources Steering Group/2019/WKLIFEIX/WKLIFE_IX_2019.pdf#page=30}{wklifeix: 4 Length-based approaches}

\href{http://ices.dk/sites/pub/Publication Reports/Expert Group Report/Fisheries Resources Steering Group/2019/WKLIFEIX/WKLIFE_IX_2019.pdf#page=58}{[wklifeix: ROC]}

\href{http://ices.dk/sites/pub/Publication Reports/Expert Group Report/Fisheries Resources Steering Group/2019/WKLIFEIX/WKLIFE_IX_2019.pdf#page=100}{[wklifeix: hake]}

The International Council of the Exploration of the Sea (ICES)** is in the process of developing methods to identify MSY proxy reference points for data-limited stocks ([WKLIFE](http://ices.dk/sites/pub/Publication Reports/Expert Group Report/Fisheries Resources Steering Group/2019/WKLIFEIX/WKLIFE_IX_2019.pdf)). The service provider is required to contribute to this process by proposing and testing new assessment models and methods of establishing reference points and will be expected to attend up to 4 one-week meetings at ICES headquarters in Copenhagen. However there are key differences with the ICES approach. Since this research contract will include stocks not currently assessed by ICES; focusing on the available data for each stock first and on the methods second; the ICES approach focuses on the methods first and then applies a limited number of methods to a large number of stocks.


